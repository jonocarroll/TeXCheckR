\documentclass[11pt,a4paper,twoside]{carrollthesis}

\usepackage{epsfig}
\usepackage{cite}
%\usepackage{graphicx}
\usepackage{rotfloat}
\usepackage{amsmath}
\usepackage{theorem}
\usepackage{amssymb}
\usepackage{latexsym}
\usepackage{epic}
\usepackage{graphics}
\usepackage{rotating}
\usepackage{placeins}
\usepackage{hyperref}
\usepackage{lscape}
%\usepackage{pdfpages}
%\usepackage{yfonts}
%\usepackage[T1]{fontec}
%%%%%%%%\usepackage[final]{pdfpages}

\newcommand{\be}{\begin{equation}}
\newcommand{\ee}{\end{equation}}
\newcommand{\fm}{\,{\rm fm}}
\newcommand{\Dslash}{\mbox{$\not \!\! D$}}
\newcommand{\delslash}{ \mkern-6mu \not \mkern-3mu \partial }
\newcommand{\del}{\partial}
\newcommand{\pr}{Phys.\ Rev.\ }
\newcommand{\physrl}{Phys.\ Rev.\ Lett.\ }
\newcommand{\physl}{Phys.\ Lett.\ }
\newcommand{\beq}{\begin{equation}}
\newcommand{\eeq}{\end{equation}}
\newcommand{\beqa}{\begin{eqnarray*}}
\newcommand{\eeqa}{\end{eqnarray*}}
\newcommand{\bea}{\begin{eqnarray}}
\newcommand{\eea}{\end{eqnarray}}
\newcommand{\bra}{\langle}
\newcommand{\ket}{\rangle}
\newcommand{\Tr}{{\rm Tr}\,}
\newcommand{\tr}{{\rm Tr}\,}
\newcommand{\muhat}{\hat{\mu}}
\newcommand{\qhat}{\hat{q}}
\newcommand{\s}{\sigma}
\newcommand{\w}{\omega}
\newcommand{\fh}{c_1(u\bar{u}+d\bar{d})}
\newcommand{\sh}{c_2(s\bar{s})}
\newcommand{\half}{\frac{1}{2}}
\newcommand{\reci}[1]{\frac{1}{#1}}
\newcommand{\dbar}{d\mkern-6mu\mathchar'26}
\newcommand{\name}{\mbox{Jonathan Carroll}}
\newcommand{\volint}[1]{\int \frac{d^4{#1}}{(2\pi)^4} \;}
\newcommand{\volthree}[1]{\int_0^{#1_F} \frac{d^3{#1}}{(2\pi)^3} \;}
\newcommand{\kslash}{\mbox{$\not \! k$}}
\newcommand{\pslash}{\mbox{$\not \! p$}}

\newcommand{\emdash}{\hspace{1pt}---\hspace{1pt}}

\newcommand{\nucl}[3]{
\ensuremath{
\phantom{\ensuremath{^{#1}_{#2}}}
\llap{\ensuremath{^{#1}}}
\llap{\ensuremath{_{\rule{0pt}{.55em}#2}}}
\mbox{#3}
}
}

% \newfont{\tapsmall}{tap scaled 1000}
% \newfont{\tap}{tap scaled 1200}
% \newfont{\tapsub}{tap scaled 1400}
% \newfont{\tapmed}{tap scaled 2200}
% \newfont{\tapbig}{tap scaled 3000}


%\input GoudyIn.fd
%\input Acorn.fd
%\newcommand*\initfamily{\usefont{U}{GoudyIn}{xl}{n}}
%\newcommand*\initfamily{\usefont{U}{Acorn}{xl}{n}} 

%\input RoyalIn.fd
%\newcommand*\initfamily{\usefont{U}{RoyalIn}{xl}{n}}


\def\qed {{%        set up
   \parfillskip=0pt        % so \par doesnt push \square to left
   \widowpenalty=10000     % so we dont break the page before \square
   \displaywidowpenalty=10000  % ditto
   \finalhyphendemerits=0  % TeXbook exercise 14.32
  %
  %                 horizontal
   \leavevmode             % \nobreak means lines not pages
   \unskip                 % remove previous space or glue
   \nobreak                % don't break lines
   \hfil                   % ragged right if we spill over
   \penalty50              % discouragement to do so
   \hskip.2em              % ensure some space
   \null                   % anchor following \hfill
   \hfill                  % push \square to right
   $\blacksquare$%              % the end-of-proof mark
  %
  %                   vertical
   \par}}                  % build paragraph

%\iffalse
\hypersetup{
%    bookmarks=true,         % show bookmarks bar?
%    unicode=false,          % non-Latin characters in Acrobat's bookmarks
%    pdftoolbar=true,        % show Acrobat's toolbar?
%    pdfmenubar=true,        % show Acrobat's menu?
%    pdffitwindow=true,      % page fit to window when opened
%    pdftitle={My title},    % title
%    pdfauthor={Author},     % author
%    pdfsubject={Subject},   % subject of the document
%    pdfnewwindow=true,      % links in new window
%    pdfkeywords={keywords}, % list of keywords
    colorlinks=true,       % false: boxed links; true: colored links
    linkcolor=red,          % color of internal links
    citecolor=green,        % color of links to bibliography
    filecolor=magenta,      % color of file links
    urlcolor=blue           % color of external links
}
%\fi

%Added by JDC
%\parindent=0mm
%\parskip=2mm

\begin{document}
% \input{frontpage}


%% Main text
% set page number starts from 1
\pagenumbering{arabic}
\setcounter{page}{1}
\pagestyle{mystyle}
% \include{Chapter1_intro}
% \FloatBarrier
% \include{Chapter2_particlephysics}
% \FloatBarrier
% \include{Chapter3_models}
% \FloatBarrier
% \include{Chapter4_methods}
% \FloatBarrier
% \include{Chapter5_results}
% \FloatBarrier
% \include{Chapter6_conclusions}
% \FloatBarrier
%%\include{intro}
%\FloatBarrier
%%\include{JLABfile}
%\FloatBarrier
%%\include{BasicIdeas}
%\FloatBarrier
%%\include{ChangeBaryons}
%\FloatBarrier
%%\include{ChemicalPotential}
%\FloatBarrier
%%\include{FandD}
%%\include{Interactions}
%%\include{NotInQHD}
%%\include{OctetQHD}
%%\include{ParticleProperties}
%%\include{SEMF}
%%\include{SU2toSU3}

% \cleardoublepage
% \addcontentsline{toc}{chapter}{\quad \ Bibliography}
% \bibliographystyle{JHEP-2}
% \bibliography{refs}
% \cleardoublepage
%\bibliographystyle{plain}
%\bibliographystyle{apsrev}
%\bibliography{refs}

% \appendix
% \FloatBarrier
% \include{AppendixA_derivations}
% \FloatBarrier
% \include{AppendixB_particles}
%\FloatBarrier
%\include{AppendixC_misc}
% \FloatBarrier
% \include{AppendixD_papers}
%%\include{FirstLawDeriv}
%%\include{QHDderiv}
%%\include{TOVderiv}
%%\include{QHDderivation_delta}
%\FloatBarrier
%%%%%%%%%\includepdf[pages=-,lastpage=22]{JDCphasetrans}
%\include{AppendixTransformation}
%\FloatBarrier
%\include{AppendixJack}
%\include{AppendixCovariance}
%\FloatBarrier
%\include{AppendixStaticQuark}
%\include{AppendixQuarkLevel}

% include a copy of my paper:
%\includepdf[pages=1-12]{CW10067_published.pdf}

\chapter{Introduction} \label{sec:introduction}
%

\font\dropcapfnt=cmr12 scaled \magstep5
%\font\dropcapfnt=frcursive10 scaled \magstep5

\def\dropcap#1{\smallbreak \setbox0=\hbox{\dropcapfnt#1}
 \dimen0=\ht0 \setbox1=\hbox{t} \advance\dimen0 by-\ht1
 \dimen1=\wd0 \advance\dimen1 by 0.3em
 \hangindent\dimen1 \hangafter-2
 \noindent\smash{\llap{\lower\dimen0\box0\hskip0.3em}}%%
  \kern-0.0em }

\dropcap The world we live in is a strange and wonderful place. The vast
majority of our interactions with objects around us can be described
and predicted by relatively simple equations and relations, which have
been fully understood for centuries now. The concept of this tactile
world is however, a result of averaging a vast ensemble of smaller
effects on an unimaginably small scale, all conspiring together to
produce what we observe at the macroscopic level. In order to
investigate this realm in which particles can fluctuate in and out of
existence in the blink of an eye, we must turn to more sophisticated
descriptions of the players on this stage.\par
%
This research has the goal of furthering our understanding of the
interior of what are loosely called `neutron stars', though as shall
be shown herein, the contents are not necessarily only neutrons. With
this in mind, further definitions include `{\it hyperon stars}', `{\it
  quark stars}', and `{\it hybrid stars}', to describe compact stellar
objects containing hyperons\footnote{Baryons for which one or more of
  the three valence quarks is a strange quark. For example, the
  $\Sigma^+$ hyperon contains two up quarks and a strange quark. For a
  table of particle properties, including quark content, refer to
  Table~\ref{tab:particlesummary}}, quarks, and a mixture of each,
  respectively. In order to do this, we require physics beyond that
  which describes the interactions of our daily lives; we require
  physics that describes the individual interactions between
  particles, and physics that describes the interactions between
  enormous quantities of particles.\par
%
Only by uniting the physics describing the realm of the large and that
of the small can one contemplate so many orders of magnitude in scale;
from individual particles with a diameter of less than $10^{-22}~{\rm
  m}$, up to neutron stars with a diameter of tens of kilometers. Yet
the physics at each end of this massive scale are unified in this
field of nuclear matter in which interactions of the smallest
theorized entities conspire in such a way that densities equivalent to
the mass of humanity compressed to the size of a mere sugar cube
become commonplace, energetically favourable, and stable.\par
%
The sophistication of the physics used to describe the world of the
tiny and that of the enormous has seen much development over time. The
current knowledge of particles has reached a point where we are able
to make incredibly precise predictions about the properties of single
particles and have them confirmed with equally astonishing accuracy
from experiments. The physics describing neutron stars has progressed
from relatively simple (yet sufficiently consistent with experiment)
descriptions of neutron (and nucleon) matter to many more
sophisticated descriptions involving various species of baryons,
mesons, leptons and even quarks.\par
%
The outcome of work such as this is hopefully a better understanding
of matter at both the microscopic and macroscopic scales, as well as
the theory and formalism that unites these two extremes. The primary
methods which we have used to construct models in this thesis are
Quantum Hadrodynamics (QHD)\emdash which shall be described in
Sec.~\ref{sec:qhd}\emdash and the Quark-Meson Coupling
(QMC) model, which shall be described in Sec.~\ref{sec:qmc}.\par
%
In this thesis, we will outline the research undertaken in which we
produce a model for neutron star structure which complies with current
theories for dense matter at and above nuclear density and is
consistent with current data for both finite nuclei and observed
neutron stars. Although only experimental evidence can successfully
validate any theory, we hope to convey a framework and model that
possesses a minimum content of inconsistencies or unjustified
assumptions such that any predictions that are later shown to be
fallacious can only be attributed to incorrect initial conditions.\par
%
As a final defense of any inconsistencies that may arise between this
research and experiment, we refer the reader to one of the author's
favourite quotes:\par
%
\vspace{0.2cm}
%
\begin{quotation}
{\raisebox{-2mm}{\Huge ``}\ 
There is a theory which states that if ever
anyone discovers exactly what the Universe is for and
why it is here, it will instantly disappear and be replaced by
something even more bizarre and inexplicable.\par \phantom{abc}}
\vspace{0.4mm}
{There is another theory which states that this 
has already happened.\ \raisebox{-2mm}{\Huge ''}}
%\vspace{-0.28cm}
\begin{flushright}
-- Douglas Adams, The Hitchhiker's Guide To The Galaxy.
\end{flushright}
\end{quotation}
%
\par
%
Our calculations will begin at the particle interaction level, which
will be described in more detail in Section~\ref{sec:PPQFT}, from
which we are able to reproduce the bulk properties of matter at high
densities, and which we shall discuss in
Sections~\ref{sec:qhd}--\ref{sec:fockterms}. The methods for producing
our simulations of the interactions and bulk properties will be
detailed in Section~\ref{sec:methods}, along with a discussion on how
this is applied to the study of compact stellar objects. The results
of the simulations and calculations will be discussed in detail in
Section~\ref{sec:results}, followed by discussions on the
interpretation of these results in Section~\ref{sec:conclusions}. For
the convenience of the reader, and for the sake of completeness,
derivations for the majority of the equations used herein are provided
in Appendix~\ref{sec:derivations}, and useful information regarding
particles is provided in Appendix~\ref{sec:particleprops}. For now
however, we will provide a brief introduction to this field of
study.\par

\section{The Four Forces}\label{sec:fourforces}
%
Theoretical particle physics has seen much success and found many
useful applications; from calculating the individual properties of
particles to precisions that rival even the best experimental setups,
to determining the properties of ensembles of particles of greater and
greater scale, and eventually to the properties of macroscopic objects
as described by their constituents.\par
%
In order to do this, we need to understand each of the four
fundamental forces in Nature. The weakest of these forces\emdash
gravity\emdash attracts any two masses, and will become most important
in the following section. Slightly stronger is electromagnetism; the
force responsible for electric charge and magnetism. This force
provides an attraction between opposite electric charges (and of
course, repulsion between like charges), and thus helps to bind
electrons to nuclei. The mathematical description of this effect is
Quantum Electrodynamics (QED).\par
%
The `weak nuclear force'\emdash often abbreviated to `the weak
force'\emdash is responsible for the decay of particles and thus
radioactivity in general. At high enough energies, this force is
unified with electromagnetism into the `electro-weak force'. The
strongest of all the forces is the `strong nuclear force', abbreviated
to `the strong force'. This force is responsible for attraction
between certain individual particles over a very short scale, and is
responsible for the binding of protons within nuclei which would
otherwise be thrown apart by the repulsive electromagnetic force
between the positively charged protons. Each of these forces plays a
part in the work contained herein, but the focus of our study will be
the strong force.\par
%
A remarkably successful description of the strongest force at the
microscopic level is `Quantum Chromodynamics' (QCD) which is widely
believed to be the true description of strong interactions, relying on
quark and gluon degrees of freedom. The major challenge of this theory
is that at low energies it is non-perturbative\footnote{At high
  energies, QCD becomes asymptotically free~\cite{Gross:1973ju} and
  can be treated perturbatively. The physics of our world however is
  largely concerned with low energies.}, in that the coupling
constant\emdash which one would normally perform a series-expansion in
powers of\emdash is large, and thus is not suitable for such an
expansion. Regularisation techniques have been produced to create
perturbative descriptions of QCD, but perturbative techniques fail to
describe both dynamical chiral symmetry breaking and confinement; two
properties observed in Nature.\par
%
Rather than working directly with the quarks and gluons of QCD,
another option is to construct a model which reproduces the effects of
QCD using an effective field theory. This is a popular method within
the field of nuclear physics, and the route that has be taken for this
work. More precisely, we utilise a balance between attractive and
repulsive meson fields to reproduce the binding between fermions that
the strong interaction is responsible for.
%
\section{Neutron Stars}\label{sec:neutronstarhistory}
%
Although gravity may be the weakest of the four fundamental forces
over comparable distance scales, it is the most prevalent over
(extremely) large distances. It is this force that must be overcome
for a star to remain stable against collapse. Although a description
of this force that unifies it with the other three forces has not been
(satisfactorily) found, General Relativity has proved its worth for
making predictions that involve large masses.\par
%
At the time when neutron stars were first proposed by Baade and
Zwicky~\cite{Baade:1934}, neutrons had only been very recently proven
to exist by Chadwick~\cite{Chadwick:1932ma}. Nonetheless, ever
increasingly more sophisticated and applicable theories have
continually been produced to model the interactions that may lead to
these incredible structures; likely the most dense configuration
of particles that can withstand collapse.\par
%
The current lack of experimental data for neutron stars permits a wide
variety of
models~\cite{Lattimer:2000nx,Heiselberg:1999mq,Weber:2004kj,SchaffnerBielich:2004ch,Weber:1989hr,Chin:1974sa},
each of which is able to successfully reproduce the observed
properties of neutron stars, and most of which are able to reproduce
current theoretical and experimental data for finite nuclei and
heavy-ion collisions~\cite{Danielewicz:2002pu,Worley:2008cb}. The
limits placed on models from neutron star
observations~\cite{Podsiadlowski:2005ig,Grigorian:2006pu,Klahn:2006iw}
do not sufficiently constrain the models, so we have the opportunity
to enhance the models based on more sophisticated physics, while still
retaining the constraints above.\par
%
The story of the creation of a neutron star begins with a reasonably
massive star, with a mass greater than eight solar masses
($M>8~M_\odot$). After millions to billions of years or so (depending
on the exact properties of the star), this star will have depleted its
fuel by fusion of hydrogen into
%helium-3, helium-4
${}^3$He, ${}^4$He, and larger elements up to iron (the most stable
element since has the highest binding energy per nucleon).\par
%
At this point, the core of the star will consist of solid iron, as the
heaviest elements are gravitationally attracted to the core of the
star, with successively lighter elements layered on top in accordance
with the traditional onion analogy. The core is unable to become any
more stable via fusion reactions and is only held up against
gravitational collapse by the degeneracy pressure of the
electrons\footnote{In accordance with the Pauli Exclusion Principle,
  no two fermions can share the same quantum state. This limits how
  close two fermions\emdash in this case, electrons\emdash can be
  squeezed, leading to the degeneracy pressure.}. The contents of the
upper layers however continue to undergo fusion to heavier elements
which also sink towards the core, adding to the mass of the lower
layers and thus increasing the gravitational pressure below.\par
%
This causes the temperature and pressure of the star to increase,
which encourages further reactions in the upper levels. Iron continues
to pile on top of the core until it reaches the Chandrasekhar limit of
$M = 1.4~M_\odot$, at which point the electron degeneracy pressure is
overcome. The next step is not fully understood\footnote{At present,
  models of supernova production have been unable to completely
  predict observations.}, but the result is a Type II supernova.\par
%
At the temperatures and pressures involved here, it is energetically
favourable for the neutrons to undergo $\beta$-decay into
protons, electrons (or muons), and antineutrinos according to
%
\be \label{eq:inversebeta}
n \to p^+ + e^- + \bar{\nu}_{e^-}.
\ee
%
These antineutrinos have a mean-free path of roughly
10~cm~\cite{Lattimer:2004} at these energies, and are therefore
trapped inside the star, causing a neutrino pressure bubble with
kinetic energy of order $10^{51}~{\rm erg} = 6.2\times 10^{56}~{\rm
  MeV}$~\cite{Lattimer:2004}. With the core collapsing (and producing
even more antineutrinos) even the rising pressure of the bubble cannot
support the mass of the material above and the upper layers begin
falling towards the core.\par
%
The sudden collapse causes a shock-wave which is believed to
`\emph{bounce}' at the core and expel the outer layers of the star in
a mere fraction of a second, resulting in what we know as a supernova,
and leaving behind the expelled material which, when excited by
radiation from another star, can be visible from across the galaxy as
a supernova remnant (SNR).\par
%
At the very centre of the SNR, the remaining core of the star
(na\"ively a sphere of neutrons, with some fraction of protons,
neutrons and electrons) retains the angular momentum of the original
star, now with a radius on the order of 10~km rather than $10^9$~km
and thus neutron stars are thought to spin very fast, with rotational
frequencies of up to 0.716~MHz~\cite{Hessels:2006ze}. Via a mechanism
involving the magnetic field of the star, these spinning neutron stars
may produce a beam of radiation along their magnetic axis, and if that
beam happens to point towards Earth to the extent that we can detect
it, we call the star a pulsar. For the purposes of this research, we
shall assume the simple case that the objects we are investigating are
static and non-rotating. Further calculations can be used to
extrapolate the results to rotating solutions, but we shall not focus
on this aspect here.\par
%
A further option exists; if the pressure and temperature (hence
energy) of the system become great enough, other particles can be
formed via weak reactions; for example, hyperons. The methods employed
in this thesis have the goal of constructing models of matter at
super-nuclear densities, and from these, models of neutron stars. The
outcome of these calculations is a set of parameters which describe a
neutron star (or an ensemble of them). Of these, the mass of a neutron
star is an observable quantity. Other parameters, such as radius,
energy, composition and so forth are unknown, and only detectable via
higher-order (or proxy) observations.\par
%
The ultimate goal would be finding a physically realistic model based
on the interactions of particles, such that we are able to deduce the
structure and global properties of a neutron star based only on an
observed mass. This however\emdash as we shall endeavor to show\emdash
is easier said than done.\par
%

\chapter{Particle Physics $\&$ Quantum Field Theory}\label{sec:PPQFT}
%  
In our considerations of the models that follow we wish to explore
ensembles of particles and their interactions. In order to describe
these particles we rely on Quantum Field Theory (QFT), which
mathematically describes the `rules' these particles obey. The
particular set of rules that are believed to describe particles
obeying the strong force at a fundamental level is Quantum
Chromodynamics (QCD), but as mentioned in the introduction, this
construction is analytically insolvable, so we rely on a model which
simulates the properties that QCD predicts.\par
%
In the following sections, we will outline the methods of calculating
the properties of matter from a field theoretic perspective.
%
\section{Lagrangian Density}\label{sec:lagrangiandensity}
%
The first step to calculating any quantity in a Quantum Field Theory
is to construct a Lagrangian density, which summarizes the dynamics of
the system, and from which the equations of motion can be
calculated. In order to do this, we must define precisely what it is
that we wish to calculate the properties of.\par
%
The classification schemes of particle physics provide several
definitions into which particles are identified, however each of these
provides an additional piece of information about those particles. We
wish to describe nucleons $N$ (consisting of protons $p$, and neutrons
$n$) which are hadrons\footnote{Bound states of quarks. In particular,
  bound states of three `valence' quarks plus any number of
  quark-antiquark pairs (the `sea' quarks, which are the result of
  particle anti-particle production via gluons) are called baryons.},
and are also fermions\footnote{Particles which obey Fermi--Dirac
  statistics, in which the particle wavefunction is anti-symmetric
  under exchange of particles; the property which leads to the Pauli
  Exclusion Principle.}.\par
%
We will extend our description to include the hyperons $Y$ (baryons
with one or more valence strange quarks) consisting of $\Lambda$,
$\Sigma^-$, $\Sigma^0$, $\Sigma^+$, $\Xi^-$, and $\Xi^0$ baryons. The
hyperons, together with the nucleons, form the octet of baryons (see
Fig.~\ref{fig:BaryonOctet}).\par
%
We can describe fermions as four-component spinors $\psi$ of
plane-wave solutions to the Dirac Equation (see later), such that
%
\be
\psi = u(\vec{p}\, ) e^{-ip_\mu x^\mu},
\ee
%
where $u(\vec{p}\, )$ are four-component Dirac spinors related to
plane-waves with wave-vector $\vec{p}$ that carry the spin information
for a particle, and which shall be discussed further in
Appendix~\ref{sec:qhdderiv}. For convenience, we can group the baryon
spinors by isospin group, since this is a degree of freedom that will
become important. For example, we can collectively describe nucleons
as a (bi-)spinor containing protons and neutrons, as
%
\be \psi_N = \begin{pmatrix} \psi_p(s) \\ \psi_n(s')
\end{pmatrix}.
%= \begin{pmatrix} \psi_p(\uparrow) \\ \psi_p(\downarrow)
%  \\ \psi_n(\uparrow) \\ \psi_n(\downarrow)
%\end{pmatrix}.
\ee
%
Here we have used the labels for protons and neutrons rather than
explicitly using a label for isospin. We will further simplify this by
dropping the label for spin, and it can be assumed that this label is
implied. We will also require the Dirac Adjoint to describe the
antibaryons, and this is written as
%
\be \bar{\psi} = \psi^\dagger \gamma^0.  \ee
%
\par
%
Similarly, we can construct spinors for all the baryons. With these
spinors we can construct a Lagrangian density to describe the
dynamics of these particles. Since we are describing spin-$\half$
particles we expect the spinors to be solutions of the Dirac equation
which in natural units (for which $\hbar = c = 1$) is written as
%
\be
\label{eq:diraceq}
\left(i\gamma^\mu\del_\mu - M\right)\psi =
\left(i\delslash-M\right)\psi = 0, \ee
%
and similarly for the antiparticle $\bar{\psi}$. Feynman slash
notation is often used to contract and simplify expressions, and is
simply defined as $\not \!\!\! A = \gamma_\mu A^\mu$. Here, $\del_\mu$
is the four-derivative, $M$ is the mass of the particle, and
$\gamma^\mu$ are the (contravariant) Dirac Matrices, which due to the
anti-commutation relation of
%
\be \label{eq:cliff} \left\{\gamma^\alpha,\gamma^\beta\right\} =
\gamma^\alpha\gamma^\beta - \gamma^\beta\gamma^\alpha =
2\eta^{\alpha\beta}\mathbb{I}, \ee
%
(where $\eta = {\rm diag}(+1,-1,-1,-1)$ is the Minkoswki metric)
generate a matrix representation of the Clifford Algebra ${\it
  Cl}(1,3)$. They can be represented in terms of the $2\times 2$
identity matrix ${\mathbb I}$, and the Pauli Matrices $\vec{\sigma}$,
as
%
\be \gamma^0 =
\left(\begin{matrix}\mathbb{I}&0\\0&-\mathbb{I}\end{matrix}\right),
\qquad \gamma^i = \left(\begin{matrix}0&\sigma^i\;
  \\-\sigma^i&0\; \end{matrix}\right).  \ee
%
\par
%
Eq.~(\ref{eq:diraceq}) describes free baryons, so we can use this as
the starting point for our Lagrangian density, and thus if we include
each of the isospin groups, we have
%
\be {\cal L} = \sum_{k} \bar{\psi}_k\left(i\delslash-M_k\right)\psi_k
\ ; \ k \in\{N,\Lambda,\Sigma,\Xi\}, \ee
%
where the baryon spinors are separated into isospin groups, as
%
\be \label{eq:isospingroups} \psi_N = \begin{pmatrix}\psi_p\\\psi_n\end{pmatrix}, \quad
  \psi_\Lambda = \begin{pmatrix}\psi_\Lambda\end{pmatrix}, \quad
    \psi_\Sigma
    = \begin{pmatrix}\psi_{\Sigma^+}\\\psi_{\Sigma^0}\\\psi_{\Sigma^-}\end{pmatrix},
    \quad \psi_\Xi
    = \begin{pmatrix}\psi_{\Xi^0}\\\psi_{\Xi^-}\end{pmatrix}. \quad
    \ee
%
This implies that the mass term is also a diagonal matrix. In many
texts this term is simply a scalar mass term multiplied by a suitable
identity matrix, but that would imply the existence of a charge
symmetry; that the mass of the proton and of the neutron were
degenerate, and exchange of charges would have no effect on the
Lagrangian density. We shall not make this assumption, and will rather
work with the physical masses as found in Ref.~\cite{Amsler:2008zzb},
so $M_k$ will contain distinct values along the diagonal.\par
%
To this point, we have constructed a Lagrangian density for the
dynamics of free baryons. In order to simulate QCD, we require
interactions between baryons and mesons to produce the correct
phenomenology.  Historically, the scalar-isoscalar
meson\footnote{Despite it's dubious status as a distinct particle
  state, rather than a resonance of $\pi\pi$.} $\s$ and
vector-isoscalar meson $\w$ have been used to this end. Additionally,
the vector-isovector $\rho$ meson has been included (for asymmetric
matter) to provide a coupling to the isospin
channel~\cite{Serot:1984ey}.\par
%
In order to describe interactions of the baryons with mesons, we can
include terms in the Lagrangian density for various classes of mesons
by considering the appropriate bilinears that each meson couples
to. For example, if we wish to include the $\w$ meson, we first
observe that as a vector meson it will couple to a vector bilinear (to
preserve Lorentz invariance) as
%
\be -ig_{\w}\bar{\psi}\gamma_\mu\w^\mu\psi \ee
%
with coupling strength $g_\w$, which as we shall see, may be dependent
on the baryon that the meson is coupled to. The particular
coefficients arise from the Feynman rules for meson-baryon vertices
(refer to Appendix~\ref{sec:diagrams}). This particular vertex is
written in Feynman diagram notation as shown in
Fig.~\ref{fig:vectorvertex}(b).\par
%
\begin{figure}[!b]
% \centering \includegraphics[width=0.65\textwidth]{IMAGES/Vertices.eps}
\caption[Baryon-meson vertices]{Interaction vertex for the (a) scalar
  and (b) vector mesons, where the solid lines represent baryons
  $\psi$, the dashed line represents a scalar meson (e.g. $\s$), and
  the wavy line represents a vector meson (e.g. $\omega_\mu$).
  \protect\label{fig:vectorvertex}}
\end{figure}
%
This is not the only way we can couple a meson to a baryon. We should
also consider the Yukawa couplings of mesons to baryons with all
possible Lorentz characteristics; for example, the $\omega$ meson can
couple to a baryon $\psi$, with several different vertices:
%
\be \bar{\psi}\gamma_\mu\omega^\mu\psi,\ \ \bar{\psi}\s_{\mu\nu}q^\nu \!
\omega^\mu\psi,\ {\rm and}\ \ \bar{\psi}q_\mu\omega^\mu\psi, \ee
%
where $q_\mu$ represents the baryon four-momentum transfer $(q_f -
q_i)_\mu$. The latter two of these provides a vanishing contribution
when considering the mean-field approximation (which shall be defined
in Section~\ref{sec:mfa}), since $\s_{00}= 0$ and $q_\mu = 0$, as the
system is on average, static.\par
%
If we include the appropriate scalar and vector terms\emdash including
an isospin-coupling of the $\rho$ meson\emdash in our basic Lagrangian
density we have
%
\be \label{eq:interL} {\cal L} = \sum_{k}
\bar{\psi}_k\left(\gamma_\mu\left[i\del^\mu-g_{k \w}\w^\mu
  -g_{\rho}(\vec{\tau}_{(k)}\cdot\vec{\rho}^{\, \mu})\; \right]
-M_k+g_{k\s}\s\right)\psi_k \ ; \ k \in\{N,\Lambda,\Sigma,\Xi\}.  \ee
%
\par
%
The isospin matrices $\vec{\tau}_{(k)}$ are scaled Pauli matrices of
appropriate order for each of the isospin groups, the third components
of which are given explicitly here as
%
\be \label{eq:taus} \tau_{(N)3} = \tau_{(\Xi)3} = \reci{2}
\left[ \begin{matrix}\ 1\ & 0\ \\\ 0\ & -1\ \end{matrix} \right],
\quad
%\tau_{(\Lambda)3} = \left[ \begin{matrix}\ 0\ \ \end{matrix} \right], \quad
\tau_{(\Lambda)3} = 0, \quad \tau_{(\Sigma)3} =
\left[ \begin{matrix}\ 1\ &\ 0\ &\ 0\ \\\ 0\ &\ 0\ &\ 0\ \\\ 0\ &\ 0\ &
    -1\ \end{matrix} \right], \ee
%
for which the diagonal elements of $\tau_{(k)3}$ are the isospin
projections of the corresponding baryons within an isospin group
defined by Eq.~(\ref{eq:isospingroups}), i.e. $\tau_{(p)3} = I_{3p} =
+\half$.\par
%
At this point it is important that we remind the reader that the
conventions in this field do not distinguish the use of explicit
Einstein summation, and that within a single equation, indices may
represent summation over several different spaces. To make this
clearer, we will show an example of a term where all the indices are
made explicit; the interaction term for the $\rho$ meson in
Eq.~(\ref{eq:interL}) for which we explicitly state all of the indices
%
\bea \nonumber {\cal L}_{\rho} &=& \sum_{k}\, g_{\rho}\, \bar{\psi}_k
\gamma^\mu \vec{\tau}_{(k)}\cdot\vec{\rho\, }_\mu \psi_k\\ &=&
\sum_k\, \sum_{i,j=1}^{f_k}\, \sum_{\alpha,\beta=1}^{4}\,
\sum_{\mu=0}^{3}\, \sum_{a=1}^{3}\, g_{\rho}\,
\left(\bar{\psi}^i_k\right)_\alpha\; (\gamma^\mu)_{\alpha \beta}\;
\left(\tau_{(k)}^a\right)^{ij}\; \rho^a_\mu\;
\left(\psi_k^{j}\right)_\beta,\eea
%
where here $k$ is summed over isospin groups $N$, $\Lambda$, $\Sigma$,
and $\Xi$; $i$ and $j$ are summed over flavor space (within an isospin
group of size $f_k$, e.g. $f_N = 2$, $f_\Sigma = 3$); $\alpha$ and
$\beta$ are summed over Dirac space; $\mu$ is summed over Lorentz
space; and $a$ is summed over iso-vector space. The Pauli matrices, of
which $\left(\tau_k^a\right)^{ij}$ are the elements, are defined in
Eq.~(\ref{eq:taus}). This level of disambiguity is overwhelmingly
cluttering, so we shall return to the conventions of this field and
leave the indices as implicit.\par
%
In addition to the interaction terms, we must also include the free
terms and field tensors for each of the mesons, which are chosen with
the intent that applying the Euler--Lagrange equations to these terms
will produce the correct phenomenology, leading to
%
\bea \nonumber &\mathcal{L} = & \sum_k
\bar{\psi}_k\left(\gamma_\mu\left[i\del^\mu-g_{k\w}\w^\mu
  -g_{\rho}(\vec{\tau}_{(k)}\cdot\vec{\rho}^{\, \mu})\; \right]
-M_k+g_{k\s}\s\right)\psi_k \\ \nonumber &&
+\ \frac{1}{2}(\partial_{\mu}\sigma\partial^{\mu}\sigma-m^{2}_{\sigma}\sigma^{2})
+\frac{1}{2}m^{2}_{\omega}\omega_{\mu}\omega^{\mu}+
\frac{1}{2}m^{2}_{\rho}\rho_{\mu}\rho^{\mu}% \\ &&
-\frac{1}{4}\Omega_{\mu\nu}\Omega^{\mu\nu} -
\frac{1}{4}R^a_{\mu\nu}R_a^{\mu\nu}, \\ && \label{eq:fulllag} \eea
%
where the field tensors for the $\w$ and $\rho$ mesons are,
respectively,
%
\be \Omega_{\mu\nu} = \del_\mu\w_\nu - \del_\nu\w_\mu, \quad
R^a_{\mu\nu} = \del_\mu\rho^a_\nu - \del_\nu\rho^a_\mu - g_\rho
\epsilon^{abc}\rho^b_\mu\rho^c_\nu.  \ee
%
This is the Lagrangian density that we will begin with for the models
we shall explore herein. Many texts (for example,
Refs.~\cite{Alaverdyan:2009kv,Greco:2000dt,Uechi:2006pz,Menezes:2002cq})
include higher-order terms (${\cal O}(\s^3)$,\ ${\cal
  O}(\s^4)$,\ $\ldots$) and have shown that these do indeed have an
effect on the state variables, but in the context of this work, we
shall continue to work at this order for simplicity. It should be
noted that the higher order terms for the scalar meson can be included
in such a way as to trivially reproduce a framework consistent with
the Quark-Meson Coupling model that shall be described later, and thus
we are not entirely excluding this contribution.\par
%

\section{Mean-Field Approximation}\label{sec:mfa}
%
To calculate properties of matter, we will use an approximation to
simplify the quantities we need to evaluate. This approximation, known
as a Mean-Field Approximation (MFA) is made on the basis that we can
separate the expression for a meson field $\alpha$ into two parts: a
constant classical component, and a component due to quantum
fluctuations;
%
\be \alpha = \alpha_{\rm classical} + \alpha_{\rm quantum}.  \ee
%
If we then take the vacuum expectation value (the average value in the
vacuum) of these components, the quantum fluctuation term vanishes,
and we are left with the classical component
%
\be \bra\alpha\ket \equiv \bra\alpha_{\rm classical}\ket.  \ee
%
This component is what we shall use as the meson contribution, and we
will assume that this contribution (at any given density) is
constant. This can be thought of as a background `field' on top of
which we place the baryon components. For this reason, we consider the
case of \emph{infinite matter}, in which there are no boundaries to
the system. The core of lead nuclei (composed of over 200 nucleons)
can be thought of in this fashion, since the effects of the outermost
nucleons are minimal compared to the short-range strong nuclear
force.\par
%
Furthermore, given that the ground-state of matter will contain some
proportion of proton and neutron densities, any flavor-changing meson
interactions will provide no contribution in the MFA, since the
overlap operator between the ground-state $|\Psi\ket$ and any other
state $|\xi\ket$ is orthogonal, and thus
%
\be
\label{eq:overlap}
\bra\; \Psi\; |\; \xi\; \ket = \delta_{\Psi\xi}.
\ee
%
For this reason, any meson interactions which, say, interact with a
proton to form a neutron will produce a state which is not the ground
state, and thus provides no contribution to the MFA. We will show in
the next section that this is consistent with maintaining isospin
symmetry.

\section{Symmetries}\label{sec:symmetries}
%
In the calculations that will follow, there are several terms that we
will exclude from our considerations {\it ab initio} (including for
example, some that appear in Eq.~(\ref{eq:fulllag})) because they
merely provide a vanishing contribution, such as the quantum
fluctuations mentioned above. These quantities shall be noted here,
along with a brief argument supporting their absence in further
calculations.\par
%
\subsection{Rotational Symmetry and Isospin}\label{subsec:rotational}
%
The first example is simple enough; we assume rotational invariance of
the fields to conserve Lorentz invariance. In order to maintain
rotational invariance in all frames, we require that the spatial
components of vector quantities vanish, leaving only temporal
components. For example, in the MFA the vector-isoscalar meson
four-vector $\w_\mu$ can be reduced to the temporal component $\w_0$,
and for notational simplicity, we will often drop the subscript and
use $\bra\alpha\ket$ for the $\alpha$ meson mean-field
contribution.\par
%
A corollary of the MFA is that the field tensor for the rho meson
vanishes;
%
\be R^a_{\mu\nu} = \del_\mu\rho^a_\nu - \del_\nu\rho^a_\mu - g_\rho
\epsilon^{abc}\rho^b_\mu\rho^c_\nu \underset{\rm MFA}\longrightarrow
R^a_{00} = 0, \ee
%
since the derivatives of the constant terms vanish and
$(\vec{\rho}_0\times\vec{\rho}_0) = 0$. The same occurs for the omega
meson field tensor
%
\be \Omega_{\mu\nu} = \del_\mu\omega_\nu - \del_\nu\omega_\mu
\underset{\rm MFA}\longrightarrow \Omega_{00} = 0.  \ee
%
We also require rotational invariance in isospin space along a
quantization direction of $\hat{z}=\hat{3}$ (isospin invariance) as
this is a symmetry of the strong interaction, thus only the neutral
components of an isovector have a non-zero contribution. This can be
seen if we examine the general $2\times2$ unitary isospin
transformation, and the Taylor expansion of this term
%
\be \psi(x) \to \psi^\prime(x) = e^{i\vec{\tau}\cdot \vec{\theta}/2}
\psi(x)
%\xrightarrow[{}_{\theta\to 0}]{} 
\ \mathop{\longrightarrow}_{|\theta| \ll 1} \ \left(1 +
i\vec{\tau}\cdot\vec{\theta}/2 \right)\psi(x), \ee
%
where $\vec{\theta}=(\theta_1,\theta_2,\theta_3)$ is a triplet of real
constants representing the (small) angles to be rotated through, and
$\vec{\tau}$ are the usual Pauli matrices as defined in
Eq.~(\ref{eq:taus}). As for the $\rho$ mesons, we can express the
triplet as linear combinations of the charged states, as
%
\be \vec{\rho} = (\rho_1,\rho_2,\rho_3) = \left(
\reci{\sqrt{2}}(\rho_+ + \rho_-),\ \frac{i}{\sqrt{2}}(\rho_- -
\rho_+),\ \rho_0 \right).  \ee
%
The transformation of this triplet is then
%
\be \vec{\rho}\; (x) \to \vec{\rho\; }^\prime(x) =
e^{i\vec{T}\cdot\vec{\theta} } \vec{\rho}\; (x).  \ee
%
where $(T^i)_{jk} = -i\epsilon_{ijk}$ is the adjoint representation of
the SU(2) generators; the spin-1 Pauli matrices in the isospin basis,
a.k.a. SO(3). We can perform a Taylor expansion about
$\vec{\theta}=\vec{0}$, and we obtain
%
\be \rho_j(x) 
%\xrightarrow[{}_{\theta\to 0}]{} 
\ \mathop{\longrightarrow}_{|\theta| \ll 1} \ 
\left[\delta_{jk} + i
  (T^i)_{jk}\theta_i\right]\rho_k(x). \ee
%
We can therefore write the transformation as
%
\be \vec{\rho}\; (x) 
%\xrightarrow[{}_{\theta\to 0}]{} 
\ \mathop{\longrightarrow}_{|\theta| \ll 1} \ \vec{\rho}\; (x) -
\vec{\theta} \times \vec{\rho}\; (x).  \ee
%
Writing this out explicitly for the three isospin states, we
obtain the individual transformation relations
%
\bea \nonumber
&&\vec{\rho}\; (x) \to \vec{\rho}\; (x) -
  \vec{\theta} \times \vec{\rho}\; (x) =
  ( \rho_1 - \theta_2\rho_3 + \theta_3\rho_2,\ \rho_2 - \theta_1\rho_3
  + \theta_3\rho_1,\ \rho_3 - \theta_1\rho_2 + \theta_2\rho_1 ).\\[2mm]
&&
\eea
%
If we now consider the rotation in only the $\hat{z}=\hat{3}$
direction, we see that the only invariant component is $\rho_3$
%
\be \vec{\rho}\; (x) 
\xrightarrow[{}_{\stackrel{\theta1 = 0}{\theta_2 = 0}}]{} 
\left(\rho_1 + \theta_3\rho_2,\ \rho_2 +
\theta_3\rho_1,\ \rho_3\right).  \ee
%
If we performed this rotation along another direction\emdash i.e. $\hat{1}$,
 $\hat{2}$, or a linear combination of directions\emdash we would find
that the invariant component is still a linear combination of charged
states. By enforcing isospin invariance, we can see that the only
surviving $\rho$ meson state will be the charge-neutral state $\rho_3
\equiv \rho_0$.\par
%
\subsection{Parity Symmetry}\label{subsec:parity}
%
We can further exclude entire isospin classes of mesons from
contributing since the ground-state of nuclear matter (containing
equal numbers of up and down spins) is a parity eigenstate, and thus
the parity operator ${\cal P}$ acting on the ground-state produces
%
\be {\cal P}|{\cal O}\ket = \pm |{\cal O}\ket.  \ee
%
Noting that the parity operator is idempotent (${\cal P}^2 = {\mathbb
  I}$), inserting the unity operator into the ground-state overlap
should produce no effect;
%
\be \bra{\cal O}|{\cal O}\ket = \bra{\cal O}|{\mathbb I}|{\cal O}\ket
= \bra{\cal O}|{\cal P}{\cal P}|{\cal O}\ket = \bra{\cal
  O}|(\pm)^2|{\cal O}\ket = \bra{\cal O}|{\cal
  O}\ket.  \ee
%
We now turn our attention to the parity transformations for various
bilinear combinations that will accompany meson interactions. For
Dirac spinors $\psi(x)$ and $\bar{\psi}(x)$ the parity transformation
produces
%
\be \begin{array}{cc} \mathcal{P}\psi(t,\vec{x}\, )\mathcal{P} =
  \gamma^0\psi(t,-\vec{x}\, ), \\[2mm]
  \mathcal{P}\bar{\psi}(t,\vec{x}\, )\mathcal{P} =
  \bar{\psi}(t,-\vec{x}\, )\gamma^0, \end{array} \ee
%
where we have removed the overall phase factor $\exp(i\phi)$ since
this is unobservable and can be set to unity without loss of
generalisation. We can also observe the effect of the parity
transformation on the various Dirac field bilinears that may appear
in the Lagrangian density. The five possible Dirac bilinears are:
%
\be \bar{\psi}\psi,\quad \bar{\psi}\gamma^\mu\psi,\quad
i\bar{\psi}[\gamma^\mu,\gamma^\nu]\psi,\quad
\bar{\psi}\gamma^\mu\gamma^5\psi,\quad i\bar{\psi}\gamma^5\psi, \ee
%
for scalar, vector, tensor, pseudo-vector and pseudo-scalar meson
interactions respectively, where $\gamma_5$ is defined as
%
\be \label{eq:gam5} \gamma_5 = i\gamma_0\gamma_1\gamma_2\gamma_3 =
%\left(\begin{array}{cccc}0&0&1&0\\0&0&0&1\\1&0&0&0\\0&1&0&0\end{array}\right),
\left(\begin{array}{cc}0&{\mathbb I}\\{\mathbb I}&0\end{array}\right),
  \ee
%
in the commonly used Dirac basis. By acting the above transformation
on these bilinears we obtain a result proportional to the spatially
reversed wavefunction $\psi(t,-\vec{x}\, )$,
%
\bea \mathcal{P}\bar{\psi}\psi \mathcal{P} &=&
+\bar{\psi}\psi(t,-\vec{x}\, ),\\[3mm]
\mathcal{P}\bar{\psi}\gamma^\mu\psi \mathcal{P} &=& \left\{
   \begin{array}{ll}
     +\bar{\psi}\gamma^\mu\psi(t,-\vec{x}\, ) &\quad \mbox{for\ }
     \mu=0,\\ -\bar{\psi}\gamma^\mu\psi(t,-\vec{x}\, ) &\quad \mbox{for\ }
     \mu=1,2,3,
   \end{array}
\right.\\[3mm] \mathcal{P}\bar{\psi}\gamma^\mu\gamma^5\psi
\mathcal{P} &=& \left\{
   \begin{array}{ll}
     -\bar{\psi}\gamma^\mu\gamma^5\psi(t,-\vec{x}\, ) &\quad \mbox{for\ }
     \mu=0,\\ +\bar{\psi}\gamma^\mu\gamma^5\psi(t,-\vec{x}\, ) &\quad
     \mbox{for\ } \mu=1,2,3,
   \end{array}
\right.\\[3mm] \mathcal{P}i\bar{\psi}\gamma^5\psi \mathcal{P} &=&
-i\bar{\psi}\gamma^5\psi(t,-\vec{x}\, ).
\eea
%
By inserting the above pseudo-scalar and pseudo-vector bilinears into
the ground-state overlap as above, and performing the parity
operation, we obtain a result equal to its negative, and so the
overall expression \emph{must} vanish. For example
%
\be \langle \mathcal{O} | i\bar{\psi}\gamma^5\psi | \mathcal{O}
\rangle = \langle \mathcal{O} | \mathcal{P} i\bar{\psi}\gamma^5\psi
\mathcal{P} | \mathcal{O} \rangle = \langle \mathcal{O} |
-i\bar{\psi}\gamma^5\psi | \mathcal{O} \rangle 
%\stackrel{!}{=} 
= 0. \ee
%
Thus all pseudo-scalar and pseudo-vector meson contributions\emdash
such as those corresponding to $\pi$ and $K$\emdash provide no
contribution to the ground-state in the lowest order. We will show
later in Chapter~\ref{sec:fockterms} that mesons can provide higher
order contributions, and the pseudo-scalar $\pi$ mesons are able to
provide a non-zero contribution via Fock terms, though we will not
calculate these contributions here.\par
%

\section{Fermi Momentum}\label{sec:kf}
%
Since we are dealing with fermions that obey the Pauli Exclusion
Principle\footnote{That no two fermions can share a single quantum
  state.}, and thus Fermi--Dirac statistics\footnote{The statistics of
  indistinguishable particles with half-integer spin. Refer to
  Appendix~\ref{sec:chempotderivn}.}, there will be restrictions on the
quantum numbers that these fermions may possess. When considering
large numbers of a single type of fermion, they will each require a
unique three-dimensional momentum $\vec{k}$ since no two fermions may
share the same quantum numbers.\par
%
For an ensemble of fermions we produce a `Fermi sea' of particles; a
tower of momentum states from zero up to some value `at the top of the
Fermi sea'. This value\emdash the Fermi momentum\emdash will be of
considerable use to us, thus it is denoted $k_F$.\par
%
Although the total baryon density is a useful control parameter, many of
the parameters of the models we wish to calculate are dependent on the
density via $k_F$. The relation between the Fermi momentum and the
total density is found by counting the number of momentum states in a
spherical volume up to momentum $k_F$ (here, this counting is performed
in momentum space). The total baryon density\emdash a number density
in units of baryons/fm$^{3}$, usually denoted as just ${\rm
  fm}^{-3}$\emdash is simply the sum of contributions from individual
baryons, as
%
\be \label{eq:rho} \rho_{\rm total} = \sum_i \rho_i = \sum_i
\frac{(2J_i +1)}{(2\pi)^3}\int \theta(k_{F_i}-|\vec{k}|) \, d^3k =
\sum_i \frac{k_{F_i}^3}{3\pi^2}, \ee
%
where here, $i$ is the set of baryons in the model, $J_i$ is the spin
of baryon $i$ (where for the leptons and the octet of baryons, $J_i =
\half$), and $\theta$ is the Heaviside step function defined as
%
\be \label{eq:Heaviside} \theta(x) =
\left\{ \begin{array}{ll}1,\ \ {\rm if} & x > 0\\0,\ \ {\rm if} & x <
  0\end{array} \right. , \ee
%
which restricts the counting of momentum states to those between $0$
and $k_F$.\par
%
We define the species fraction for a baryon $B$, lepton $\ell$, or
quark $q$ as the density fraction of that particle, denoted by $Y_i$,
such that
%
\be 
\label{eq:Y}
Y_i = \frac{\rho_i}{\rho_{\rm total}}\ ;\quad i\in \{
%p,n,\Lambda,\Sigma^+,\Sigma^0,\Sigma^-,\Xi^0,\Xi^-,\ell,q 
B,\ell,q\} \, . \ee
%
Using this quantity we can investigate the relative proportions of
particles at a given total density.
%

\section{Chemical Potential}\label{sec:qftchempot}
%
In order to make use of statistical mechanics we must define the some
important quantities. One of these will be the chemical potential
$\mu$, also known as the Fermi energy $\epsilon_{F}$; the energy of a
particle at the top of the Fermi sea, as described in
Appendix~\ref{sec:chempotderivn}. This energy is the relativistic
energy of such a particle, and is the energy associated with a Dirac
equation for that particle. For the simple case of a non-interacting
particle, this is
%
\be \label{eq:muepsf} \mu_B = \epsilon_{F_B} = \sqrt{k_{F_B}^2 +
  M_B^2}. \ee
%
In the case that the baryons are involved in interactions with mesons,
we need to introduce scalar and temporal self-energy terms, which (for
example) for Hartree-level QHD using a mean-field approximation are
given by
%
\be \label{eq:selfenergy}
\Sigma^s_B = - g_{B\s} \bra\s\ket,\quad \Sigma^0_B = g_{B\w}\bra\w\ket
+ g_{\rho}I_{3B}\bra\rho\ket,
\ee
%
where $I_{3B}$ is the isospin projection of baryon $B$, defined by the
diagonal elements of Eq.~(\ref{eq:taus}), and where the scalar
self-energy is used to define the baryon effective mass as
%
\be \label{eq:effM}
M_B^* = M_B + \Sigma^s_B = M_B - g_{B\s} \bra\s\ket,
\ee
%
These self-energy terms affect the energy of a Dirac equation, and
thus alter the chemical potential, according to
%
\be
\label{eq:mu_sw}
\mu_B = \sqrt{k_{F_B}^2 + (M_B + \Sigma^s_B)^2}+\Sigma_B^0\ .  \ee
%
Eq~(\ref{eq:effM}) and Eq.~(\ref{eq:mu_sw}) define the important
in-medium quantities, and the definition of each will become
dependent on which model we are using.\par
%
For a relativistic system such as that which will consider here, each
conserved quantity is associated with a chemical potential, and we can
use the combination of these associated chemical potentials to obtain
relations between chemical potentials for individual species. In our
case, we will consider two conserved quantities: total baryon number
and total charge, and so we have a chemical potential related to each
of these. We can construct the chemical potential for each particle
species by multiplying each conserved charge by its associated
chemical potential to obtain a general relation. Thus
%
\be \label{eq:chempotrel} \mu_i = B_i \mu_n - Q_i \mu_e, \ee
%
where; $i$ is the particle species (which can be any of the baryons)
for which we are constructing the chemical potential; $B_i$ and $Q_i$
are the baryon number (`baryon charge', which is unitless) and
electric charge (normalized to the proton charge) respectively; and
$\mu_n$ and $\mu_e$ are the chemical potentials of neutrons and
electrons, respectively. Leptons have $B_\ell=0$, and all baryons have
$B_B=+1$. The relations between the chemical potentials for the octet
of baryons are therefore derived to be
%
\be \label{eq:allmus}
\begin{array}{rcrcrcl}
\mu_\Lambda &=& \mu_{\Sigma^0} &=& \mu_{\Xi^0} &=& \mu_n, \\
&&\mu_{\Sigma^-} &=& \mu_{\Xi^-} &=& \mu_n + \mu_e, \\
&&\mu_p &=& \mu_{\Sigma^+} &=& \mu_n - \mu_e, \\
&&&& \mu_\mu &=& \mu_e.
\end{array}
\ee
%
A simple example of this is to construct the chemical potential for
the proton (for which the associated charges are $B_p = +1$ and $ Q_p
= +1$);
%
\be \label{eq:betaeq} \mu_p = \mu_n - \mu_e.  \ee
%
This can be rearranged to a form that resembles neutron $\beta$-decay
%
\be
\label{eq:n0beta}
\mu_n = \mu_p + \mu_e.  \ee
%
\par
%
If we were to consider further conserved charges, such as lepton
number for example, we would require a further associated chemical
potential. In that example, the additional chemical potential would be
for (anti)neutrinos $\mu_{\bar{\nu}}$. The antineutrino would be
required to preserve the lepton number on both sides of the equation;
the goal of such an addition. Since we shall consider that neutrinos
are able to leave the system considered, we can ignore this
contribution {\it ab inito}. The removal of this assumption would
alter Eq.~(\ref{eq:n0beta}) to include the antineutrino, as would
normally be expected in $\beta$-decay equations
%
\be \mu_n = \mu_p + \mu_e + \mu_{\bar{\nu}}.  \ee
%

\section{Explicit Chiral Symmetry (Breaking)}\label{sec:chiral}
%
One of the most interesting symmetries of QCD is chiral symmetry. If
we consider the QCD Lagrangian density to be the sum of quark and
gluon contributions, then in the massless quark limit ($m_q = 0$);
%
%%%%%%%%%%%%%%%%%%%%
%\be
%\label{eq:QCDm0lag}
%{\cal L}_{\rm QCD} = {\cal L}_{g}+{\cal L}_{q}
%= -\frac{1}{4g^2} {\rm Tr}\left[G_{\mu\nu} G^{\mu\nu}\right] +
%i\sum_{f=1}^{N_f} \bar{\psi}_f D_{\mu}\gamma^{\mu} \psi_f, 
%\ee
%
%where here, $D_\mu = (\del_\mu - igA_\mu)$ is the covariant
%derivative, $g$ is the QCD coupling, $A_\mu$ are the gluon fields,
%$G_{\mu\nu} = \left(\left[\del_\mu,A^a_\nu\right]
%-gf^{abc}A_\mu^bA_\nu^c\right)$ is the gluon field strength tensor,
%$f^{abc}$ are the SU(3) structure constants, and $\psi_f$ is the quark
%spinor for flavor $f \in \{u,d,s,\ldots ,N_f\}$.\par
%%%%%%%%%%%%%%%%%%%%
%
\bea
\nonumber
{\cal L}_{\rm QCD} &=& {\cal L}_g + {\cal L}_q \\
\nonumber
&=& -\reci{4}G_{\mu\nu}^aG_{a}^{\mu\nu}
+ \bar{\psi}_i i \gamma^\mu (D_\mu)_{ij} \psi_j
\\
\label{eq:QCDm0lag}
&=& -\reci{4}G_{\mu\nu}^aG_{a}^{\mu\nu} +
\bar{\psi}_i i\gamma^\mu\del_\mu \psi_i -
gA^a_\mu\bar{\psi}_i\gamma^\mu T_{ij}^a\psi_j\ , \eea 
%$T^a \equiv  are the generators of SU(3)
where here, $\psi_i(x)$ is a quark field of color $i \in \{r,g,b\}$,
$A_\mu^a(x)$ is a gluon field with color index $a \in \{1,\ldots,8\}$,
$T_{ij}^a$ is a generator\footnote{For example, $T^a = \lambda^a/2$
  using the Hermitian Gell-Mann matrices $\lambda_a$.} for SU(3), $g$
is the QCD coupling constant, and $G_{\mu\nu}^a$ represents the
gauge-invariant gluonic field strength tensor, given by
%
\be
G_{\mu\nu}^a = \left[\del_\mu,A_\nu^a\right] - gf^{abc}A_\mu^b
A_\nu^c\ ,
\ee
%
written with the structure constants $f^{abc}$.
%
%If we consider i.e. $N_f = 2$ for now, we can group together the
%flavour spinors in a single quark spinor
%%
%\be
%\label{eq:udspinors}
%\psi = \begin{pmatrix}\psi_u\\\psi_d\end{pmatrix}.
%\ee
%%
Left- and right-handed components of Dirac fields can be
separated using the projection operators
%
\be \psi_{L\atop R} = \frac{1\mp \gamma_5}{2}\psi, \ee
%
using the definition of $\gamma_5$ of Eq.~(\ref{eq:gam5}), and so the
quark terms in the QCD Lagrangian density (the gluon terms are not
projected) can be written in terms of these components as
%
\be {\cal L}_{q}^{(f)} = i\bar{\psi}^{(f)}_L D_{\mu}\gamma^{\mu}
\psi^{(f)}_L + i\bar{\psi}^{(f)}_R D_{\mu}\gamma^{\mu} \psi^{(f)}_R.
\ee
%
This Lagrangian density is invariant under rotations in U(1) of the left- and
right-handed fields
%
\bea &{\rm U}(1)_L:& \psi_L \to e^{i\alpha_L}\psi_L, \quad
\psi_R \to \psi_R, \\ &{\rm U}(1)_R:& \psi_R \to e^{i\alpha_R}\psi_R,
\quad \psi_L \to \psi_L, \eea
%
where $\alpha_L$ and $\alpha_R$ are arbitrary phases. This invariance
is the chiral ${\rm U}(1)_L\otimes {\rm U}(1)_R$ symmetry. The Noether
currents associated with this invariance are then
%
\be J_L^\mu = \bar{\psi_L}\gamma^\mu \psi_L, \qquad J_R^\mu =
\bar{\psi_R}\gamma^\mu \psi_R, \ee
%
and as expected, these currents are conserved, such that $\del_\mu
J_L^\mu = \del_\mu J_R^\mu = 0$ according to the Dirac Equation. These
conserved currents can be alternatively written in terms of conserved
vector and axial-vector currents, as
%
\be J^\mu_{L} = \frac{V^\mu - A^\mu}{2}, \qquad J^\mu_{R} =
\frac{V^\mu + A^\mu}{2}, \ee
%
where here, $V^\mu$ and $A^\mu$ denote the vector and axial-vector
currents respectively\emdash the distinction of $A^\mu$ here from the
gluon fields in Eq.~(\ref{eq:QCDm0lag}) is neccesary\emdash and these
are defined by
%
\be V^\mu = \bar{\psi}\gamma^\mu\psi,\qquad A^\mu =
\bar{\psi}\gamma^\mu\gamma_5\psi, \ee 
%
and which are also conserved, thus $\del_\mu V^\mu = \del_\mu A^\mu =
0$. The chiral symmetry of ${\rm U}(1)_L\otimes {\rm U}(1)_R$ is
therefore equivalent to invariance under transformations under ${\rm
  U}(1)_V \otimes {\rm U}(1)_A$, where we use the transformations
%
\bea
\label{eq:U1V}
&{\rm U}(1)_V:& \psi \to e^{i\alpha_V}\psi,\qquad \bar{\psi} \to \psi^\dagger
e^{-i\alpha_V} \gamma_0,\\
\label{eq:U1A}
&{\rm U}(1)_A:& \psi \to e^{i\alpha_A\gamma_5}\psi,\quad \ \bar{\psi} \to
\psi^\dagger e^{-i\alpha_A\gamma_5} \gamma_0.
\eea
%
\par
%
Using the anticommutation relation
%
\be \label{eq:anticommutation} \left\{\gamma_5,\gamma_\mu\right\} =
\gamma_5\gamma_\mu + \gamma_\mu\gamma_5 = 0 \ee
% 
we can evaluate the effect that the vector and axial-vector
transformations have on the QCD Lagrangian density, and we find that both
transformations are conserved.
%
\iffalse 
The chiral symmetry is defined by the chiral transformation
as
%
\be
\label{eq:chiral}
\psi^\prime = {\cal U}\psi; \quad {\cal U} =
\exp\left(-ig_5\gamma_5\half\tau_i\epsilon_i\right); \quad {\cal U}
\in {\rm SU}(2) \ee
%
in which $\vec{\tau}$ operates in isospin- (flavor-) space, $\gamma_5$
operates in Dirac-space, and where $g_5$ and $\vec{\epsilon}$ are
parameters of the rotations in their respective spaces. Here,
$\gamma_5$ is defined as
%
\be \gamma_5 = i\gamma_0\gamma_1\gamma_2\gamma_3 =
\left(\begin{array}{cccc}0&0&1&0\\0&0&0&1\\1&0&0&0\\0&1&0&0\end{array}\right).
  \ee
%
${\cal U}$ is a unitary matrix (${\cal U}^\dagger{\cal U} = I$), and because of the
anticommutation between $\gamma_5$ and $\gamma_\mu$;
%
\be
\left\{\gamma_5,\gamma_\mu\right\} = \gamma_5\gamma_\mu -
\gamma_\mu\gamma_5 = 0,
\ee
%
we have the relation
%
\be
\gamma_\mu {\cal U} = {\cal U}^\dagger \gamma_\mu.
\ee
%
Using this, we can write the transformation on the spinors as
%
\bea
\nonumber
&\psi^\prime &= {\cal U}\psi\\
&\bar{\psi}^\prime &= \psi^\dagger{\cal U}^\dagger\gamma_0 =
\psi^\dagger\gamma_0{\cal U} = \bar{\psi}{\cal U}
\eea
%
If we now observe the effect of this transformation on the bilinear
form in Eq.~(\ref{eq:QCDm0lag}) 
%
\be
\bar{\psi}^\prime\Dslash\psi^\prime 
= \bar{\psi}{\cal U}\gamma_\mu D^\mu{\cal U}\psi
= \bar{\psi}\gamma_\mu D^\mu {\cal U}^\dagger {\cal U}\psi
= \bar{\psi}\Dslash\psi,
\ee
%
we can clearly see that this is invariant under the chiral
transformation, Eq.~(\ref{eq:chiral}).\par
%
Provided that the quarks are massless, this symmetry is exactly
conserved and the QCD Lagrangian density is symmetric with respect to
rotations in the flavor space independently for right- and left-handed
quarks. QCD therefore has the global symmetry described by the group
${\rm SU}(N_{f})_R\times {\rm SU}(N_{f})_L$.\par
%
\fi
%
If we now consider a quark mass term ${\cal L}_m$ in the QCD
Lagrangian density, the fermionic part becomes
%
\be {\cal L}^\psi_{\rm QCD} = {\cal L}_q + {\cal L}_m = 
%\sum_{f=1}^{N_f}
\bar{\psi}_i\left(i\gamma^\mu
(D_\mu)_{ij}-m\delta_{ij}\right)\psi_j. \ee
%
%where $M$ is a diagonal matrix of quark masses. 
For the purposes of these discussions, we can set the masses of the
quarks to be equal without loss of generality.
%, and thus $m_f = m \ \forall \ f$. 
Although the massless Lagrangian density possesses both of the above
symmetries, the axial vector symmetry\emdash and hence chiral
symmetry\emdash is explicitly broken by this quark mass term;
%
\be
\label{eq:mbreaking}
{\cal L}_{m} = -\bar{\psi}m \psi \stackrel{{\rm U}(1)_A}{\longrightarrow}
-\bar{\psi}m e^{2i\alpha_A}\psi \neq -\bar{\psi}m \psi. 
\ee
%
The vector symmetry is nonetheless preserved when including this
term.\par

\section{Dynamical Chiral Symmetry (Breaking)}\label{sec:dynamicchiral}
%
Even with a massless Lagrangian density, it is possible that chiral
symmetry becomes dynamically broken, and we refer to this as
Dynamically Broken Chiral Symmetry, or DCSB.\par
%
Following the description of Ref.~\cite{Roberts:1994dr}, if we
consider the basic Lagrangian density of QCD to be
%
\be {\cal L}_{\rm QCD} = \bar{\psi}_i\left(i\gamma^\mu (D_\mu)_{ij} -
m \delta_{ij}\right) \psi_j - \reci{4}G_{\mu\nu}^aG^{\mu\nu}_a, \ee
%
with the definitions as in the previous section, of
%
\be
G^a_{\mu\nu} = \del_\mu A^a_\nu - \del_\nu A^a_\mu - g
f^{abc}A^b_\mu A^c_\nu, \quad 
D_\mu = \del_\nu + i g A^a_\mu T^a,% = \del^\mu - ig A^\mu,
\ee
%
with standard definitions of other terms, then we can write the sum of
all QCD One-Particle Irreducible (1-PI) diagrams\footnote{Diagrams
  that cannot be made into two separate disconnected diagrams by
  cutting an internal line are called One-Particle Irreducible, or
  1-PI.} with two external legs as shown in Fig.~\ref{fig:quarkDSE};
illustrating the quark self-energy. The expression for the
renormalized quark self-energy in $d$ dimensions is
%
\be \label{eq:quarkDSE} -i\Sigma(p) = \frac{4}{3}Z_r\, g^2 \! \int
\frac{d^d q}{(2\pi)^d}
(i\gamma_\mu)(iS(q))(iD^{\mu\nu}(p-q))(i\Gamma_\nu(q,p)), \ee
%
where $Z_r$ is a renormalization constant, $g$ is the QCD
coupling, and $q$ is the loop momentum.\par
%
In the absence of matter fields or background fields (the
Lorentz-covariant case), we can write this self-energy as a sum of
Dirac-vector and Dirac-scalar components, as
%
\be \label{eq:LcovSE} \Sigma(p) = \pslash\; \Sigma_{\rm DV}(p^2) +
\Sigma_{\rm DS}(p^2).  \ee
%
where $\Sigma_{\rm DV}(p^2)$ is the Dirac-vector component, and
$\Sigma_{\rm DS}(p^2)$ is the Dirac-scalar component. These must both
be functions of $p^2$, since there are no other Dirac-fields to
contract with, and $\Sigma(p)$ is a Lorentz invariant quantity in this
case.\par
%
For the purposes of our discussion in this section, we will
approximate the Dirac-vector component of the self-energy to be
$\Sigma_{\rm DV} \sim 1$, in which case the self-energy is dependent
only on the Dirac-scalar component.\par
%
Even with a massless theory ($m = 0$) it is possible that the
renormalized self-energy develops a non-zero Dirac-scalar component,
thus $\Sigma_{\rm DS}(p^2) \neq 0$. This leads to a non-zero value for
the quark condensate $\bra\bar{\psi}_q\psi_q\ket$, and in the limit of
exact chiral symmetry, leads to the pion becoming a massless Goldstone
boson. Thus chiral symmetry can be dynamically broken. With the
addition of a Dirac-scalar component of the self-energy, the
Lagrangian density becomes
%
\be {\cal L}_{QCD} = \bar{\psi}_i\left(i\gamma^\mu (D_\mu)_{ij} - (m +
\Sigma_{\rm DS})\delta_{ij} \right) \psi_j -
\reci{4}G_{\mu\nu}^aG^{\mu\nu}_a, \ee
%
\vfill
%
\begin{figure}[!h]
\centering
% \includegraphics[width=0.65\textwidth]{IMAGES/quarkDSE.eps}
\caption[Quark self-energy (DSE) in QCD]{Feynman diagram for the QCD
  self-energy for a quark, as given by the Dyson--Schwinger Equation
  (DSE). The full expression for this is given in
  Eq.~(\ref{eq:quarkDSE}). \protect\label{fig:quarkDSE}}
\end{figure}
%
\clearpage
%\vfill
%
and we can define a dynamic quark mass via the gap equation;
%
\be m^* = m + \Sigma_{\rm DS}. \ee
%
We will continue this discussion in Section~\ref{sec:njl}, in which we
will describe a particular model for $\Sigma_{\rm DS}$ in
order to describe DCSB.\par
%
\section{Equation of State}\label{sec:EoS}
%
In order to investigate models of dense matter, we need to construct
an Equation of State (EOS), which is simply a relation between two or
more state variables\emdash those which thermodynamically describe the
current state of the system, such as temperature, pressure, volume, or
internal energy\emdash under a given set of physical conditions. With
this, we will be able to investigate various aspects of a model and
compare differences between models in a consistent fashion.\par
%
For our purposes, we use the total baryon density $\rho_{\rm total}$
as the control parameter of this system, and so we need to obtain the
connection between, say, the energy density ${\cal E}$, the pressure
$P$, and this total baryon density, i.e.
%
\be {\cal E} = {\cal E}(\rho_{\rm total}), \qquad P = P(\rho_{\rm
  total}).  \ee
%
\par
%
State variables are important quantities to consider. Within any
transition between states the total change in any state variable will
remain constant regardless of the path taken, since the change is an
exact differential, by definition. For the hadronic models described
herein, the EOS are exact, in that they have an analytic form;
%
\be \label{eq:Panalytic} P(\rho_{\rm total}) = \rho_{\rm
  total}^2\frac{\del}{\del\rho_{\rm total}}\left(\frac{{\cal
    E}(\rho_{\rm total})}{\rho_{\rm total}}\right).  \ee
%
As simple as this exact form may seem, the derivative complicates
things, and we will find it easier to calculate the pressure
independently. Nonetheless, this expression will hold true. More
interestingly, this expression is equivalent to the first law of
thermodynamics in the absence of heat transfer; i.e.
%
\be PdV = - dE, \ee
%
(the proof of which can be found in Appendix~\ref{sec:firstlaw}) which
assures us that the theory is thermodynamically consistent.\par
%
A notable feature of each symmetric matter EOS we calculate is the
effect of saturation; whereby the energy per baryon for the system
possesses a global minimum at a particular value of the Fermi
momentum. This can be considered as a binding energy of the system. In
symmetric matter (in which the densities of protons and neutrons are
equal), the nucleon Fermi momenta are related via \mbox{$k_F = k_{F_n}
  = k_{F_p}$}, and the energy per baryon (binding energy) $E$ is
determined via
%
\be \label{eq:EperA} E = \left[ \reci{\rho_{\rm total}} \left( {\cal
    E} -\sum_B \rho_BM_B\right) \right].  \ee
%
In order to reproduce (a chosen set of) experimental results, this
value should be an extremum of the curve with a value of \mbox{$E_0 =
  -15.86~{\rm MeV}$} at a density of \mbox{$\rho_0 = 0.16~{\rm
    fm}^{-3}$} (or the corresponding Fermi momentum $k_{F_0}$).\par
%
The nucleon symmetry energy $a_{\rm sym}$ is approximately a measure
of the energy difference between the energy per baryon (binding
energy) of a neutron-only model and a symmetric nuclear model
(essentially a measure of the breaking of isospin symmetry). A more
formal expression (without assuming degeneracy between nucleon
masses, as derived in Appendix~\ref{sec:symenergy}) is
%
\be \label{eq:a4}
a_{\rm sym} = \frac{g_{\rho}^2}{3 \pi^2 m_\rho^2} k_{F}^3 
+ \reci{12} \frac{k_{F}^2}{\sqrt{k_{F}^2 + (M_p^*)^2}} 
+ \reci{12} \frac{k_{F}^2}{\sqrt{k_{F}^2 + (M_n^*)^2}} \, . 
\ee
%
At saturation, this should take the value of $(a_{\rm sym})_0 =
32.5~{\rm MeV}$ (for an analysis of values, see
Ref.~\cite{Tsang:2008fd}).\par
%
Another important aspect of an EOS is the compression modulus $K$
which represents the \emph{stiffness} of the EOS; the ability to
withstand compression. This ability is intimately linked to the Pauli
Exclusion Principle in that all other things being equal, a system
with more available states (say, distinguishable momentum states) will
have a softer EOS, and thus a smaller compression modulus. The
compression modulus itself is defined as the curvature of the binding
energy at saturation, the expression for which is
%
\be \label{eq:Kmod} K = \left[ k_F^2
  \frac{d^2}{dk_F^2}\left(\frac{\cal E}{\rho_{\rm
      total}}\right)\right]_{k_{F_{\rm sat}}} = 9 \left[ \rho_{\rm
    total}^2 \frac{d^2}{d\rho_{\rm total}^2}\left(\frac{\cal
    E}{\rho_{\rm total}}\right)\right]_{\rho=\rho_0}. \ee
%
The motivation for this is that by compressing the system, the energy
per baryon will rise. The curvature at saturation determines how fast
that rise will occur, and thus how resistant to compression the system
is. Experimentally, this is linked to the properties of finite nuclei,
particularly those with a large number of nucleons, and the binding of
these within a nucleus.\par
%
According to Ref.~\cite{Serot:1984ey} this should have a value in the
range $200$--$300~{\rm MeV}$, and we will calculate the value of
$K$ for each of the models to follow for comparison.\par
%

\section{Phase Transitions}\label{sec:phasetransitions}
%
In order to consider transitions between different phases of matter we
must use statistical mechanics. The simplest method of constructing a
phase transition\emdash known as a `Maxwell transition'\emdash is an
isobaric (constant pressure) transition constructed over a finite
density range. A transition of this form remains useful in
understanding the liquid--gas style phase transition that occurs
within QHD, which is a first-order transition (similar to that of ice
melting in a fluid) with the phases being separated by a non-physical
negative-pressure region. The inclusion of a Maxwell transition to
this simple model for QHD removes this unphysical region and replaces
it with a constant pressure phase.\par
%
The method for constructing a Maxwell transition will not be covered
here, though in-depth details can be found in
Ref.~\cite{Muller:1995ji}. We can however extract the transition
densities from Ref.~\cite{Serot:1984ey} to reproduce the results,
which are shown later in Fig.~\ref{fig:EOS_plusmaxwell} for the
various varieties of QHD. The more sophisticated method of
constructing a phase transition\emdash the `Gibbs
transition'~\cite{Glendenning:2001pe} that we have used for the
results produced herein\emdash relies on a little more statistical
mechanics. A comparison between the Maxwell and Gibbs methods for
models similar to those used in this work can be found in
Ref.~\cite{Bhattacharyya:2009fg}. For a full in-depth discussion of
this topic, see Ref.~\cite{Reif}.\par
%
If we consider a homogeneous (suitable for these mean-field
calculations) system with energy $E$, volume $V$, $N_m$ particles of
type $m$, and entropy $S$ which depends on these parameters such that
%
\be S=S(E,V,N_1,\ldots,N_m), \ee
%
then we can consider the variation of the entropy in the system as a
function of these parameters, resulting in
%
\be \label{eq:dSfullform} dS = \left(\frac{\del S}{\del
  E}\right)_{V,N_1,\ldots,N_m} \ dE + \left(\frac{\del S}{\del
  V}\right)_{E,N_1,\ldots,N_m} \ dV + \sum_{i=1}^m \ \left(\frac{\del
  S}{\del N_i}\right)_{V,N_{j\neq i}} \ dN_i, \ee
%
using standard statistical mechanics notation whereby a subscript $X$
on a partial derivative $\displaystyle{\left(\del A / \del
  B\right)_{X}}$ denotes that $X$ is explicitly held constant.
Eq.~(\ref{eq:dSfullform}) should be equal to the fundamental
thermodynamic relation when the number of particles is fixed, namely
%
\be \label{eq:fundtherm} dS = \frac{\dbar Q}{T} = \frac{dE + PdV}{T}.
\ee
%
Here, the symbol $\dbar$ denotes the inexact differential, since the
heat $Q$ is not a state function\emdash does not have initial and
final values\emdash and thus the integral of this expression is only
true for infinitesimal values, and not for finite values.
%
Continuing to keep the number of each type of particle $N_i$ constant,
a comparison of coefficients between
Eqs.~(\ref{eq:dSfullform})~and~(\ref{eq:fundtherm}) results in the
following relations:
%
\be \label{eq:invTdefs} \left(\frac{\del S}{\del
  E}\right)_{V,N_i,\ldots,N_m} = \reci{T}, \quad \left(\frac{\del
  S}{\del V}\right)_{E,N_i,\ldots,N_m} = \frac{P}{T}.  \ee
%
To provide a relation similar to Eq.~(\ref{eq:invTdefs}) for the case
where $dN_i \neq 0$, one defines $\mu_j$\emdash the chemical potential
per molecule\emdash as
%
\be \mu_i = -T\left(\frac{\del S}{\del N_i}\right)_{E,V,N_{j\neq i}}.
\ee
%
We can now re-write Eq.~(\ref{eq:dSfullform}) with the definitions in
Eq.~(\ref{eq:invTdefs}) for the case where the particle number can
change, as
%
\be 
\label{eq:dS}
dS = \reci{T}dE + \frac{P}{T}dV - \sum_{i=1}^m \frac{\mu_i}{T}dN_i,
\ee
%
which can be equivalently written in the form of the fundamental
thermodynamic relation for non-constant particle number,
%
\be dE = TdS - PdV + \sum_{i=1}^m \mu_i dN_i.  \ee
%
If we now consider a system $X$ of two phases $A$ and $B$, then we can
construct relations between their parameters by considering the
following relations:
%
\bea
\nonumber
E_X &=& E_A + E_B, \\
V_X &=& V_A + V_B, \\
\nonumber
N_X &=& N_A + N_B.
\eea
%
If we consider that these quantities are conserved between phases, we
find the following conservation conditions
%
\be%a
%\nonumber 
\label{eq:changesAB}
\begin{array}{rcrcr}
dE_X = 0 &\Rightarrow &dE_A + dE_B = 0 &\Rightarrow &dE_A = - dE_B, \\[2mm]
%
dV_X = 0 &\Rightarrow &dV_A + dV_B = 0 &\Rightarrow &dV_A = - dV_B,
\\[2mm]
%\nonumber
%
dN_X = 0 &\Rightarrow &dN_A + dN_B = 0 &\Rightarrow &dN_A = - dN_B .
\end{array}
\ee%a
%
\par
%
The condition for phase equilibrium for the most probable situation is
that the entropy must be a maximum for
$S=S_X(E_X,V_X,N_X)=S(E_A,V_A,N_A;E_B,V_B,N_B)$, which leads to
%
\be dS_X = dS_A + dS_B = 0.  \ee
%
Thus, inserting Eq.~(\ref{eq:dS}) we find
%
\be dS = \left(\reci{T_A} dE_A + \frac{P_A}{T_A} dV_A -
\frac{\mu_A}{T_A} dN_A \right) + \left(\reci{T_B} dE_B +
\frac{P_B}{T_B} dV_B - \frac{\mu_B}{T_B} dN_B \right).  \ee
%
If we now apply the result of Eq.~(\ref{eq:changesAB}) we can simplify
this relation to
%
\be \label{eq:maxS} dS = 0 = \left(\reci{T_A} - \reci{T_B}\right) dE_A
+ \left(\frac{P_A}{T_A} - \frac{P_B}{T_B}\right) dV_A -\left(
\frac{\mu_A}{T_A} - \frac{\mu_B}{T_B}\right) dN_A \ee
%
and thus for arbitrary variations of $E_A$, $V_A$ and $N_A$, each
bracketed term must vanish separately, so that
%
\be \label{eq:eachvanishes} \reci{T_A} = \reci{T_B}, \quad
\frac{P_A}{T_A} = \frac{P_B}{T_B}, \quad \frac{\mu_A}{T_A} =
\frac{\mu_B}{T_B}.  \ee
%
Eq.~(\ref{eq:eachvanishes}) implies that at the phase transition the
system will be isentropic ($dS=0$), isothermal ($dT=0$), isobaric
($dP=0$), and isochemical ($d\mu=0$), where the terms
$S$,~$T$,~$V$,~and~$\mu$ now refer to the mean values rather than for
individual particles.\par
%
We only require two systems at any one time when considering a mixture
of phases, for example, a neutron (`neutron phase') can transition to
a proton and an electron (`proton and electron phase') provided that
the condition $\mu_n = \mu_p + \mu_e$ is met.\par
%
%For phase transitions we need to consider some statistical mechanics. The conditions for any phase  
%equilibrium is that the entropy must be a maximum for $S=S(E,V,N)$ which can be expanded as 
%
%\bea
%\nonumber
%&dS &= \left[ \left( \frac{\partial S_1}{\partial E_1}\right)_{V_1,N_1}-
%\left( \frac{\partial S_2}{\partial E_2}\right)_{V_2,N_2} \right] dE_1 \\
%\nonumber
%&& \ \ + \left[ \left( \frac{\partial S_1}{\partial V_1}\right)_{E_1,N_1}-
%\left( \frac{\partial S_2}{\partial V_2}\right)_{E_2,N_2} \right] dV_1 \\
%\nonumber
%&& \ \ \ + \left[ \left( \frac{\partial S_1}{\partial N_1}\right)_{E_1,V_1}-
%\left( \frac{\partial S_2}{\partial N_2}\right)_{E_2,V_2} \right] dN_1 = 0. \\
%&&  
%\label{eq:maxS}
%\eea
%%
%Eq.~(\ref{eq:maxS}) is satisfied when the temperature (and hence, energy) and pressure 
%(hence volume) are held constant ($dE = dV = 0$), and when
%%
%\be
%\label{eq:dSdN}
%\qquad \left( \frac{\partial S_1}{\partial N_1}\right)_{E_1,V_1}=
%\left( \frac{\partial S_2}{\partial N_2}\right)_{E_2,V_2},
%\ee
%%
%which implies that the chemical potentials for each phase be equal, since
%%
%\be
%\label{eq:dSdN:mu}
%\qquad \mu_i = -T_i \left( \frac{\partial S_i}{\partial N_i}\right)_{E_i,V_i} ; \quad i=1,2.
%\ee
%%
%Since chemical potential is simply the Gibbs free energy per particle,
%%
%\be
%\label{eq:mu:Gibbs}
%\qquad \mu = \frac{1}{N}G(T,P,N),
%\ee
%%
%we see that we are really considering a Gibbs condition for phase equilibrium. This is more or 
%less the justification for the Maxwell construction, though the calculations become quite complex.
%%
%The transitions that we will be most interested in is from hadronic matter to quark matter. To do this 
%however, we will require a quark matter model, which we have not decided on yet. This will provide us 
%with a model of a `hybrid star' - presumably hadronic near the surface with a quark core.
%
For a phase transition between hadronic- and quark-matter phases then,
the conditions for stability are therefore that chemical, thermal, and
mechanical equilibrium between the hadronic $H$, and quark $Q$ phases
is achieved, and thus that the independent quantities in each phase
are separately equal. Thus the two independent chemical potentials (as
described in Sec.~\ref{sec:qftchempot}) $\mu_n$~and~$\mu_e$ are each
separately equal to their counterparts in the other phase,
i.e. $\left[(\mu_n)_H=(\mu_n)_Q\right]$, and
$\left[(\mu_e)_H=(\mu_e)_Q\right]$ for chemical equilibrium;
$\left[T_H=T_Q\right]$ for thermal equilibrium; and $\left[P_H =
  P_Q\right]$ for mechanical equilibrium.\par
%
An illustrative example of these relations is shown in
Fig.~\ref{fig:3d} in which the values of the independent chemical
potentials $\mu_n$ and $\mu_e$, as well as the pressure $P$ for a
hadronic phase and a quark phase are plotted for increasing values of
total density $\rho_{\rm total}$. In this case, the quark matter data
is calculated based on the hadronic matter data, using the chemical
potentials in the hadronic phase as inputs for the quark phase
calculations, and as such the chemical potentials are\emdash by
construction\emdash equal between the phases. As this is an
illustrative example of the relations between the phases, no
constraints have been imposed to reproduce a phase transition yet.\par
%
In this figure, the low-density points correspond to small values of
$\mu_n$, and we see that for densities lower than some phase
transition density $\rho_{\rm total} < \rho_{\rm PT}$ the hadronic
pressure is greater than the quark pressure and thus the hadronic
phase is dominant. At the transition the pressures are equal, and thus
both phases can be present in a mixed phase, and beyond the transition
the quark pressure is greater than the hadronic pressure indicating
that the quark phase becomes dominant.\par
%
Note that for all values of the total density, the chemical potentials
in each phase are equal, as shown by the projection onto the
$\mu_n$$\mu_e$ plane.\par
%
In our calculations, we will only
investigate these two phases independently up to the phase transition,
at which point we will consider a mixed phase, as shall be described
in the next section.\par
%
\begin{figure}[!t]
\centering
% \includegraphics[width=0.9\textwidth]{IMAGES/paper_images/fig3.eps}
\caption[Illustrative locus of phase transition variables]{(Color
  Online) Illustrative locus of values for the independent chemical
  potentials $\mu_e$ and $\mu_n$, as well as the pressure $P$ for
  phases of hadronic matter and deconfined quark matter. Note that
  pressure in each phase increases with density. and that a projection
  onto the $\mu_n\mu_e$ plane is a single line, as ensured by the
  chemical equilibrium condition. \protect\label{fig:3d}}
\end{figure}
%
We consider both phases to be cold on the nuclear scale, and assume
$T=0$ so the temperatures are also equal, again by construction. We
must therefore find the point\emdash if it exists\emdash at which, for
a given pair of independent chemical potentials, the pressures in both
the hadronic phase and the quark phase are equal.\par
%
To find the partial pressure of any baryon, quark, or lepton species
$i$ we use
%
\be \label{eq:pressures} P_i = \frac{\left(2J_B + 1\right) {\cal
    N}_c}{3(2\pi)^3}\int \frac{\vec{k}^2\; \theta(k_{F_i} -
  |\vec{k}|)}{\sqrt{\vec{k}^2+(M_i^*)^2}}\; d^3k, \ee
%
where the number of colors is ${\cal N}_c = 3$ for quarks, ${\cal N}_c
= 1$ for baryons and leptons, and where $\theta$ is the Heaviside step
function defined in Eq.~(\ref{eq:Heaviside}). To find the total
pressure in each phase, we sum the pressures contributions in that
phase. The pressure in the hadronic phase $H$ is given by
%
%\be P = \sum_i P_i \ ; \quad i \in \{H,Q\} \ee
%
%where the pressure for the hadronic phase $H$ is
%
\be \label{eq:Hpressure}
P_H = \sum_j P_j  + \sum_\ell P_\ell + \sum_m P_m,
%+ \sum_{\alpha=\w,\rho} \reci{2} m_\alpha^2 \bra\alpha\ket^2 -
%\reci{2} m_\s^2 \bra\s\ket^2,
\ee
%
in which $j$ represents the baryons, $\ell$ represents the leptons,
$m$ represents the mesons appearing in the particular model being
considered, if they appear, and the pressure in the quark phase $Q$ is
given by
%which is equivalent to Eq.~(\ref{eq:P_H}), and
%
\be \label{eq:Qpressure}
P_Q = \sum_q P_q + \sum_\ell P_\ell - B,
\ee
%
where $q$ represents the quarks, and $B$ denotes the bag energy
density, which we shall discuss further in Section~\ref{sec:MITbag}.\par
%
%For the QMC model
%described in Section~\ref{sec:qmc}, and a Fermi gas of quarks, both
%with interactions with leptons for charge neutrality, a point exists
%at which the condition of stability, as described above, is
%satisfied.\par
%
%At this point it is equally favourable that hadronic matter and quark
%matter are the dominant phase. Beyond this point the quark pressure
%is greater than the hadronic pressure; the phase with a greater
%pressure will be favoured, and we can imagine that this phase with a
%greater pressure expells the phase with lower pressure.
%
%a system is in equilibrium when
%the grand potential $\Omega$ is a minimum, and the pressure can be
%defined as
%
%\be P=-\frac{\del\Omega}{\del V}; \quad \Omega=-N d\mu - S dT - P dV
%\ee
%
%Thus a larger pressure indicates a 
%Thus the quark phase will be more energetically favourable. 
In order to determine the EOS beyond the point at which the pressures
are equal, we need to consider the properties of a mixed phase.\par
%

\section{Mixed Phase} \label{sec:MixedPhase}

Once we have defined the requirements for a phase transition between
two phases, we must consider the possibility of a mixed phase (MP)
containing proportions of the two phases. This adds a further degree
of sophistication to a model; we can not only find equations of state
for hadronic matter and quark matter and simply stitch them together,
but we can also allow the transition between these to occur
gradually.\par
%
To calculate the mixed phase EOS, we calculate the hadronic EOS with
control parameter $\rho_{\rm total}$, and use the independent chemical
potentials $\mu_n$ and $\mu_e$ as inputs to determine the quark matter
EOS, since we can determine all other Fermi momenta given these two
quantities. We increase $\rho_{\rm total}$ until we find a
density\emdash if it exists\emdash at which the pressure in the quark
phase is equal the pressure in the hadronic phase (if such a density
cannot be found, then the transition is not possible for the given
models).\par
%
Assuming that such a transition is possible, once we have the density
and pressure at which the phase transition occurs, we change the
control parameter to the quark fraction $\chi$ (which is an order
parameter parameterizing the transition to the quark matter phase)
which determines the proportions of hadronic matter and quark
matter. If we consider the mixed phase to be composed of some fraction
of hadronic matter and some fraction of quark matter, then the mixed
phase of matter will have the following properties: the total density
will be
%
\be 
\label{eq:mp_rho}
\rho_{\rm MP} = (1-\chi)\; \rho_{\rm HP} + \chi\; \rho_{\rm QP},
\ee
%
where $\rho_{\rm HP}$ and $\rho_{\rm QP}$ are the densities in the
hadronic and quark phases, respectively. A factor of three in the
equivalent baryon density in the quark phase,
%
\be 
\label{eq:equivrho}
\rho_{\rm QP} = \reci{3}\sum_q \rho_q = (\rho_u + \rho_d + \rho_s)/3,
\ee
%
arises because of the restriction that a baryon contains three
quarks.\par
%
According to the condition of mechanical equilibrium detailed earlier,
the pressure in the mixed phase will be
%
\be
\label{eq:mp_P}
P_{\rm MP} = P_{\rm HP} = P_{\rm QP}.
\ee
%
\par
%
We can step through values $0\! <\! \chi\! <\! 1$ and determine the
properties of the mixed phase, keeping the mechanical stability
conditions as they were above. In the mixed phase we need to alter our
definition of charge neutrality; while previously we have used the
condition that two phases were independently charge-neutral, such as
$n\to p^+ + e^-$, it now becomes possible that one phase is (locally)
charged, while the other phase carries the opposite charge, making the
system globally charge-neutral. This is achieved by enforcing
%
\be \label{eq:mp_charge}
0 = (1-\chi)\; \rho^c_{\rm HP} + \chi\; \rho^c_{\rm QP} + 
\rho^c_{\ell} \, ,
\ee
%
where this time we are considering charge-densities, which are simply
the sum of densities multiplying their respective charges
%
\be \rho^c_i = \sum_j Q_j \rho_j \ ; \quad i \in \{{\rm HP},{\rm
  QP},\ell\}, \ee
%
where $j$ are the all individual particles modelled within the
grouping $i$. For example, the quark charge-density in a non-interacting
quark phase is given by
%
\be \label{eq:qp_charge} \rho^c_{\rm QP} = \sum_q Q_q \rho_q =
\frac{2}{3}\rho_u - \reci{3}\rho_d - \reci{3}\rho_s.  \ee
%
\par
%
We continue to calculate the properties of the mixed phase for
increasing values of $\chi$ until we reach $\chi = 1$, at which point
the mixed phase is now entirely charge-neutral quark matter. This
corresponds to the density at which the mixed phase ends, and a pure
quark phase begins. We can therefore continue to calculate the EOS for
pure charge-neutral quark matter, once again using $\rho_{\rm total}$
as the control parameter, but now where the total density is the
equivalent density as defined in Eq.~(\ref{eq:equivrho}).\par
%

\section{Stellar Matter} \label{sec:stellarmatter}
%
The equations of state described above are derived for homogeneous
infinite matter. If we wish to apply this to a finite system we must
investigate the manner in which large ensembles of particles are held
together. The focus of this work is `neutron stars', and we must find
a way to utilise our knowledge of infinite matter to provide insight
to macroscopic objects. For this reason, we turn to the theory of
large masses; General Relativity.\par
%
The Tolman--Oppenheimer--Volkoff (TOV)
equation~\cite{Oppenheimer:1939ne} describes the conditions of
stability against gravitational collapse for an EOS, i.e. in which the
pressure gradient is sufficient to prevent gravitational collapse of
the matter. The equations therefore relate the change in pressure with
radius to various state variables from the EOS. To preserve
continuity, the equations are solved under the condition that the
pressure at the surface of the star \emph{must} be zero.\par
%
The TOV equation is given by
%
\be
\label{eq:TOV:full}
\ \ \frac{dP}{dr}=-\frac{ G \left( P / c^2 +\mathcal{E} \right)
  \left(M(r)+4 r^3 \pi P / c^2 \right)} {r(r-2 G M(r) / c^2)}, \ee
%
or, in Planck units\footnote{In which certain fundamental physical
  constants are normalized to unity, viz $\hbar = c = G = 1$.}
%
\be
\label{eq:TOV:nat}
\qquad \frac{dP}{dr}=-\frac{ \left( P+\mathcal{E} \right) \left(M(r)
  +4 \pi r^{3} P\right)}{r(r -2M(r))}, \ee
%
where the mass within a radius $R$ is given by integrating the energy
density, as
%
\be
\label{eq:TOV:massdef}
\qquad M(R) = \int_0^R 4 \pi r^{2} {\cal E}(r) \; dr.
\ee
%
For a full derivation of these equations, refer to
Appendix~\ref{sec:tovderiv}.\par
%
Supplied with these equations and a derived EOS, we can calculate
values for the total mass and total radius of a star\footnote{Stellar
  objects in these calculations are assumed to be static, spherically
  symmetric, and non-rotating, as per the derivation of this
  equation. For studies of the effect of rapid rotation in General
  Relativity see Refs.~\cite{Lattimer:2004nj,Owen:2005fn}.} for a given
central density. We refer to these values as the `stellar
solutions.'\par
%
This is particularly interesting, since the mass of a neutron star is
observable (either via observing a pair of objects rotating about a
barycenter\footnote{A common centre of mass for the system, the point
  about which both objects will orbit, which is the balance point of
  the gravitational force. In this case, the mass measurements are
  simplified.}, or some other indirect/proxy measurement), yet the
radius is not directly observable, as stars are sufficiently distant
that they all appear as `point-sources'. With these calculations, we
produce a relationship between two quantities: the stellar mass and
the stellar radius, of which only the mass is currently observable,
and even this is not always so. This provides useful data for further
theoretical work requiring both quantities, as well as an opportunity
to place theoretical bounds on future experimental observations.\par
%
In addition to this data, since we are able to solve our equations for
the radial distance from the centre of the star, we can provide data
that current experiments can not; we can investigate the interior of a
neutron star, by calculating the proportions of various particles at
successive values of internal radius and/or density. This allows us to
construct a cross-section of a neutron star, investigate the possible
contents, and examine the effects that various changes to the models
have on both the internal and external properties.\par
%
%\clearpage 

\section{SU(6) Spin-Flavor Baryon-Meson Couplings}\label{sec:su6deriv}

We have noted earlier that the coupling of baryons to mesons is
dependent on isospin group. The physics leading to this result is
highly non-trivial, but is often neglected in the literature. We will
therefore outline the process involved in determining the relations
between baryon-meson couplings.\par
%
In order to determine the normalized relations between the point
vertex couplings of various mesons to the full baryon octet $g_{Bm}$
it is common to express the octet as a $3\times 3$ matrix in flavor
space as
%
\be B = \left( \
\begin{matrix}
\frac{\Sigma^0}{\sqrt{2}}+\frac{\Lambda}{\sqrt{6}} & \Sigma^+ & p \\
\Sigma^- & -\frac{\Sigma^0}{\sqrt{2}}+\frac{\Lambda}{\sqrt{6}} & n \\
-\Xi^- & \Xi^0 & -\frac{2 \Lambda}{\sqrt{6}}
\end{matrix} \ \right).
\ee
%
This has been constructed as an array where rows and columns are
distinguished by rotations in flavor space, which can be seen if we
observe the quark content of these baryons;
%
\be B = \left( \
\begin{matrix}
uds & \mathbf{d \to u} & uus & \mathbf{s \to d} & uud \\ \mathbf{u \to
  d} & & & & \\ dds & & dus & & ddu \\ \mathbf{d \to s} & & & & \\ ssd
& & uss & & sud
\end{matrix} \ \right).
\ee
%
The vector meson octet ($J^P=1^-$) can be written in a similar fashion as
%
\be \label{eq:Pvec}
P^\textrm{vec}_\textrm{oct} = \left( \
\begin{matrix}
\frac{\rho^0}{\sqrt{2}}+\frac{\omega_8}{\sqrt{6}} & \rho^+ & K^{*+} \\
\rho^- & -\frac{\rho^0}{\sqrt{2}}+\frac{\omega_8}{\sqrt{6}} & K^{*0} \\
K^{*-} & \overline{K^{*0}} & -2\left(\frac{\omega_8}{\sqrt{6}} \right)
\end{matrix} \ \right),
\ee
%
which, along with the singlet state, $P_{\rm sing}^{\rm vec} =
\reci{\sqrt{3}}{\rm diag}(\omega_0,\omega_0,\omega_0)$ defines the
vector meson nonet
%
\be \label{eq:octplussing} P^{\rm vec} = P^{\rm vec}_{\rm oct} +
P^{\rm vec}_{\rm sing}.  \ee
%
Furthermore, the scalar meson octet ($J^P=0^+$) can be written as
%
\be \label{eq:Psca}
P^\textrm{sca}_\textrm{oct} = \left( \
\begin{matrix}
\frac{a_0^0}{\sqrt{2}}+\frac{\s_8}{\sqrt{6}} & a_0^+ & \kappa^+ \\
a_0^- & -\frac{a_0^0}{\sqrt{2}}+\frac{\s_8}{\sqrt{6}} & \kappa^0 \\
\kappa^- & \overline{\kappa^0} & -2\frac{\s_8}{\sqrt{6}}
\end{matrix} \ \right),
\ee
%
and along with singlet state $P^{\rm sca}_{\rm sing} =
\reci{\sqrt{3}}{\rm diag}(\s_0,\s_0,\s_0)$, these define the scalar
meson nonet. The meson octet matrices are constructed in a similar
fashion to the baryon octet matrix;
%
\be
P_\textrm{oct} = \left( \
\begin{matrix}
u\bar{u} & \mathbf{\bar{u} \to \bar{d}} & u\bar{d} & \mathbf{\bar{d}
  \to \bar{s}} & u\bar{s} \\ \mathbf{u \to d} & & & & \\ d\bar{u} & &
d\bar{d} & & d\bar{s} \\ \mathbf{d \to s} & & & & \\ s\bar{u} & &
s\bar{d} & & s\bar{s}
\end{matrix} \ \right).
\ee
%
The singlet and octet representations of both $\omega$ and $\sigma$
($\w_0, \w_8, \s_0, \s_8$, appearing in
Eqs.~(\ref{eq:Pvec})--(\ref{eq:Psca})) are not however the physical
particles which we wish to include in the model; these are linear
combinations of the physical particles. Due to explicit SU(3)
flavor-symmetry breaking ($m_s > m_u,\ m_d$), a mixture of the
unphysical $\omega_8$ and $\w_0$ states produces the physical $\omega$
and $\phi$ mesons, while a mixture of the unphysical $\sigma_8$ and
$\s_0$ states produces the physical $\sigma$ and $f_0$ mesons, the
properties of which we list in Appendix~\ref{sec:particleprops}.\par
%
The octet and singlet states are represented by linear combinations of
quark-antiquark pairs. The state vectors for these are
%
\be
\label{eq:quarklincombs}
|\omega_8\ket = |\sigma_8\ket = \reci{\sqrt{6}} \left( |\bar{u}u\ket +
|\bar{d}d\ket -2 |\bar{s}s\ket \right),\quad |\omega_0\ket =
|\sigma_0\ket = \reci{\sqrt{3}} \left( |\bar{u}u\ket + |\bar{d}d\ket
|+ |\bar{s}s\ket \right), \ee
%
where the normalizations arise by ensuring that 
%
\be
\bra\xi|\xi\ket = 1;\quad \xi \in \{\omega_8,\omega_0,\sigma_8,\sigma_0\}.
\ee
%
Since the quark contents of the physical states are predominantly 
%
\be \omega = \sigma = \reci{\sqrt{2}}\ (u\bar{u}+d\bar{d}),\quad {\rm
  and} \quad \phi = f_0 = -s\bar{s}, \ee
%
we can replace the octet and singlet combinations with the physical
states via the replacements of
%
\bea
&\omega_8 = \reci{\sqrt{3}}\ \omega +
\frac{2}{\sqrt{6}}\ \phi,\quad 
&\omega_0 = \sqrt{\frac{2}{3}}\ \omega
-\reci{\sqrt{3}}\ \phi,\\
&\sigma_8 = \reci{\sqrt{3}}\ \sigma +
\frac{2}{\sqrt{6}}\ f_0,\quad 
&\sigma_0 = \sqrt{\frac{2}{3}}\ \sigma
-\reci{\sqrt{3}}\ f_0. 
\eea
%
Using these definitions, we can express the octet and singlet states 
in terms of the physical states
%
\be P^\textrm{vec}_\textrm{oct} = \left( \
\begin{matrix}
\frac{\rho^0}{\sqrt{2}}+\frac{\omega}{\sqrt{18}}+\frac{\phi}{\sqrt{9}}
& \rho^+ & K^{*+} \\ \rho^- &
-\frac{\rho^0}{\sqrt{2}}+\frac{\omega}{\sqrt{18}}+\frac{\phi}{\sqrt{9}}
& K^{*0} \\ K^{*-} & \overline{K^{*0}} &
-2\left(\frac{\omega}{\sqrt{18}}+\frac{\phi}{\sqrt{9}}\right)
\end{matrix} \ \right),
\ee
%
\be P^\textrm{sca}_\textrm{oct} = \left( \
\begin{matrix}
\frac{a_0^0}{\sqrt{2}}+\frac{\sigma}{\sqrt{18}}+\frac{f_0}{\sqrt{9}}
& a_0^+ & \kappa^{+} \\ a_0^- &
-\frac{a_0^0}{\sqrt{2}}+\frac{\sigma}{\sqrt{18}}+\frac{f_0}{\sqrt{9}}
& \kappa^{0} \\ \kappa^{-} & \overline{\kappa^{0}} &
-2\left(\frac{\sigma}{\sqrt{18}}+\frac{f_0}{\sqrt{9}}\right)
\end{matrix} \ \right).
\ee
%
Each of the above mesons can interact with a pair of baryons in three
possible SU(3) invariant ways, which we shall identify as $F$-style
(anti-symmetric), $D$-style (symmetric) and $S$-style
(singlet). The singlet mesons $\omega_0$ and $\sigma_0$ are associated
with an $S$-style coupling, while the octet particles are associated
with $F$- and $D$-style couplings.\par
%
To determine these $F$-, $D$-, and $S$-style couplings for the vector
and scalar mesons we need to calculate the SU(3) invariant Lagrangian
density coefficients symbolically for each isospin group, with each
SU(3) invariant combination\footnote{Following the notation
  of Ref.~\cite{Rijken:1998yy}.} given by:
%
\bea \nonumber \left[ \bar{B}BP \right]_F &= &{\rm Tr}(\bar{B}PB)-{\rm
  Tr}(\bar{B}BP) \\ \nonumber &= &{\rm Tr}(\bar{B}P_{\rm oct}B)-{\rm
  Tr}(\bar{B}BP_{\rm oct}), \\[2mm] \left[ \bar{B}BP \right]_D &=
&{\rm Tr}(\bar{B}PB)+{\rm Tr}(\bar{B}BP)-\frac{2}{3}{\rm
  Tr}(\bar{B}BP){\rm Tr}(P) \\ \nonumber &= &{\rm Tr}(\bar{B}P_{\rm
  oct}B)+{\rm Tr}(\bar{B}BP_{\rm oct}), \\[2mm] \nonumber \left[
  \bar{B}BP \right]_S &= &{\rm Tr}(\bar{B}B){\rm Tr}(P) \\ \nonumber
&= &{\rm Tr}(\bar{B}B){\rm Tr}(P_{\rm sing}).\label{eq:SU3invcombs}
\eea
%
where we have expanded the meson matrices $P$ according to
Eq.~(\ref{eq:octplussing}), and we note that the octet matrices
$\bar{B}$, and $B$ are traceless.\par
%
Together, these terms can be combined to form an interaction
Lagrangian density for all possible SU(3) invariant interactions
involving each meson isospin group, with octet and singlet coupling
coefficients $F$, $D$, and $S$ (each defined separately for each
isospin group) as
%
\be
\label{eq:SU3intL}
{\cal L}^{\rm int} = -\sqrt{2}\left\{ F\left[ \bar{B}BP\right]_F +D
\left[ \bar{B}BP\right]_D \right\} - S\reci{\sqrt{3}} \left[
  \bar{B}BP\right]_S, \ee
%
where the remaining numerical factors are introduced for
convenience.\par
%
If we evaluate this Lagrangian density by matrix multiplication of
$B$, the octet and singlet matrices of $P_{\rm vec}$, and $\bar{B} =
B^{\dagger}\gamma^0$ in the combinations stated in
Eq.~(\ref{eq:SU3invcombs}), we can extract the coefficients of each
baryon-meson vertex in terms of $F$, $D$ and $S$ factors. These are
summarized in Table~\ref{tab:FDScouplings} for vertices involving the
physical vector mesons $\w$ and $\rho^0$ and $\phi$, for pairs of like
baryons\footnote{As discussed in Section~\ref{sec:mfa}, any
  flavor-changing meson-baryon interactions would produce a null
  overlap of ground-state operators, and as such we only focus on the
  like-baryon interactions of the form $g_{B\alpha}\bar{\psi}_B \alpha
  \psi_{B^\prime} \delta_{BB^\prime}$ for a meson $\alpha$ in this
  discussion.}.
%the formulation of the Lagrangian densities we construct in each model
%for which we calculate properties will rely on isospin grouped
%spinors, and since we have enforced symmetries which prevent
%interactions of charged mesons (refer to Section~\ref{subsec:rotational})
%we do not include any flavour-changing interactions, and thus we
%restrict our calcultions to contributions from terms proportional to
%$\bar{\psi}_B\psi_{B^\prime} \delta_{BB^\prime}$. 
The summary for the scalar mesons is the same under replacements of
$\omega \to \sigma$, $\vec{\rho} \to \vec{a}_0$, and $\phi \to
f_0$.\par
%
\vfill
%
\begin{table}[!h]
\caption[$F$-, $D$-, and $S$-style couplings for
  $\bar{B}BP$]{\protect\label{tab:FDScouplings}$F$-, $D$-, and
  $S$-style couplings of like baryon-baryon pairs to vector mesons
  used in these models, according to vertices of type $B + P \to
  \bar{B}$. The summary for the scalar mesons is the same under the
  replacements of $\w \to \s$, $\vec{\rho} \to \vec{a}_0$, and
  $\phi\to f_0$.}
\begin{center}
\hrule
\vspace{1mm}
\hrule
\vspace{2mm}
\hspace{0.2cm}
\centering
\begin{tabular}{cc}
\begin{minipage}[c]{0.5\textwidth}
$\begin{array}{lrl}
\overline{\Sigma^-}\Sigma^-\rho^0 \ & \propto & \ 2 F \\[2mm]
\overline{\Sigma^-}\Sigma^-\omega \ & \propto & \ \frac{1}{9} \left(-6 D-\sqrt{6}S\right)           \\[2mm]
\overline{\Sigma^-}\Sigma^-\phi   \ & \propto & \ \frac{1}{9} \left(-6 \sqrt{2}D+\sqrt{3} S\right)  \\[2mm]
\overline{\Sigma^0}\Sigma^0\omega \ & \propto & \ \frac{1}{9} \left(-6 D-\sqrt{6}S\right)           \\[2mm]
\overline{\Sigma^0}\Sigma^0\phi   \ & \propto & \ \frac{1}{9} \left(-6 \sqrt{2}D+\sqrt{3} S\right)  \\[2mm]
\overline{\Sigma^+}\Sigma^+\rho^0 \ & \propto & \ -2 F \\[2mm]
\overline{\Sigma^+}\Sigma^+\omega \ & \propto & \ \frac{1}{9} \left(-6 D-\sqrt{6}S\right)           \\[2mm]
\overline{\Sigma^+}\Sigma^+\phi   \ & \propto & \ \frac{1}{9} \left(-6 \sqrt{2}D+\sqrt{3} S\right)  \\[2mm]
 \end{array}$
 \end{minipage}
 & 
 \begin{minipage}[c]{0.5\textwidth}
$\begin{array}{lrl}
\overline{\Lambda }\Lambda\omega \ & \propto & \ \frac{1}{9} \left(6 D-\sqrt{6} S\right)           \\[2mm]
\overline{\Lambda }\Lambda \phi  \ & \propto & \ \frac{1}{9} \left(6 \sqrt{2} D+\sqrt{3}S\right)     \\[2mm]
\overline{p}p\rho^0              \ & \propto & \ -D-F  \\[2mm]
\overline{p}p\omega              \ & \propto & \ \frac{1}{9} \left(3 D-9 F-\sqrt{6} S\right)  \\[2mm]
\overline{p}p\phi                \ & \propto & \ \frac{1}{9} \left(3 \sqrt{2} D-9 \sqrt{2} F+\sqrt{3}S\right)  \\[2mm]
\overline{n}n\rho^0              \ & \propto & \ D+F  \\[2mm]
\overline{n}n\omega              \ & \propto & \ \frac{1}{9} \left(3 D-9 F-\sqrt{6} S\right)  \\[2mm]
\overline{n}n\phi                \ & \propto & \ \frac{1}{9} \left(3 \sqrt{2} D-9 \sqrt{2} F+\sqrt{3}S\right)  \\[2mm]
 \end{array}$
 \end{minipage}
\end{tabular} 
 \begin{minipage}[c]{0.5\textwidth}
$\begin{array}{lrl}
& & \\
\overline{\Xi^-}\Xi^-\rho^0      \ & \propto & \ -D+F  \\[2mm]
\overline{\Xi^-}\Xi^-\omega      \ & \propto & \ \frac{1}{9} \left(3 D+9 F-\sqrt{6}S\right) \\[2mm]
\overline{\Xi^-}\Xi^-\phi        \ & \propto & \ \frac{1}{9} \left(3 \sqrt{2} D+9 \sqrt{2}F+\sqrt{3} S\right)  \\[2mm]
\overline{\Xi^0}\Xi^0\rho^0      \ & \propto & \ D-F  \\[2mm]
\overline{\Xi^0}\Xi^0\omega      \ & \propto & \ \frac{1}{9} \left(3 D+9 F-\sqrt{6}S\right)  \\[2mm]
\overline{\Xi^0}\Xi^0\phi        \ & \propto & \ \frac{1}{9} \left(3 \sqrt{2} D+9 \sqrt{2}F+\sqrt{3} S\right)  \\[2mm]
 \end{array}$
\end{minipage}
%
\vspace{1mm}
\hrule
\vspace{1mm}
\hrule
\end{center}
\end{table}
%
The physical $\phi$ and $f_0$ states are purely strange quark
components. These do not couple to nucleons significantly (since
nucleons contain only up and down valence quarks) the only way to
produce these mesons is via gluons. Thus we set the (normalized or
not) couplings of these mesons to zero; $g_{B\phi}=g_{Bf_0}=0$. We are
then left with the physical $\sigma$ and $\omega$ mesons as the
effective meson degrees of freedom.\par
%
If we denote the \emph{total} (but not normalized) coupling (now
including all prefactors of Eq.~(\ref{eq:SU3intL})) of a (like) baryon
pair $\bar{B}B$ to a meson $\alpha$ by $f_{B\alpha}$, and we calculate
the SU(3) invariant combinations for the singlet $\w_0$ and mixed
state $\w_8$, we can use the relation between these normalizations
from Eq.~(\ref{eq:quarklincombs}) to relate the $S$-style couplings to
the remaining couplings via
%
\be -\frac{S}{3} = f_{N\w_0} = \sqrt{2}f_{N\w_8} =
\frac{\sqrt{2}}{\sqrt{3}}(D-3F), \ee
%
and so we can reduce the relation of the couplings to
%
\be
\label{eq:strangecoupling}
S = \sqrt{6}(3F-D),
\ee
%
and we can therefore find the couplings of the singlet $\w_0$ meson in
terms of just $F$ and $D$ factors. After removing the strange quark
components and substituting the result of
Eq.~(\ref{eq:strangecoupling}) we obtain a summary of couplings as
shown in Table~\ref{tab:FDcouplings}.\par
%
\begin{table}[!b]
\caption[$F$- and $D$-style couplings for
  $\bar{B}BP$]{\protect\label{tab:FDcouplings}Couplings of like
  baryon-baryon pairs to vector mesons used in these models, according
  to vertices of type $B + P \to \bar{B}$ using the relation of
  Eq.~(\ref{eq:strangecoupling})}
\begin{center}
\hrule
\vspace{1mm}
\hrule
\vspace{2mm}
\hspace{0.2cm}
\centering
\begin{tabular}{ccc}
\begin{minipage}[c]{0.3\textwidth}
%$\begin{array}{rclrr}
%    &  \overline{\Sigma^-} &  \leftarrow  &   \Sigma^-   +   \rho^0  \   =&    2F                                     \\[2mm]
%    &  \overline{\Sigma^-} &  \leftarrow  &   \Sigma^-   +   \omega  \   =&    -\frac{2D}{3}                          \\[2mm]
%    &  \overline{\Sigma^0} &  \leftarrow  &  \Sigma^0    +   \omega  \   =&    -\frac{2D}{3}                 \\[2mm]
%    &  \overline{\Sigma^0} &  \leftarrow  &  \Lambda     +   \rho^0  \   =&    -\frac{2D}{\sqrt{3}}          \\[2mm]
%    &  \overline{\Sigma^+} &  \leftarrow  &  \Sigma^+    +   \rho^0  \   =&    -2   F                        \\[2mm]
%    &  \overline{\Sigma^+} &  \leftarrow  &  \Sigma^+    +   \omega  \   =&    -\frac{2D}{3}                 \\[2mm]
% \end{array}$
%%%
$\begin{array}{lrr}
\overline{\Sigma^-}\Sigma^-\rho^0  \  & \propto &    2F                    \\[2mm]
\overline{\Sigma^-}\Sigma^-\omega  \  & \propto &    -\frac{2D}{3}         \\[2mm]
\overline{\Sigma^0}\Sigma^0\omega  \  & \propto &    -\frac{2D}{3}         \\[2mm]
\overline{\Sigma^+}\Sigma^+\rho^0  \  & \propto &    -2   F                \\[2mm]
\overline{\Sigma^+}\Sigma^+\omega  \  & \propto &    -\frac{2D}{3}         \\[2mm]
 \end{array}$
 \end{minipage}
 & 
 \begin{minipage}[c]{0.3\textwidth}
% $  \begin{array}{rclrr}
%     &  \overline{p}        &  \leftarrow  &   p          +   \rho^0  \   =&    -D-F                                   \\[2mm]
%     &  \overline{p}        &  \leftarrow  &   p          +   \omega  \   =&    \frac{D}{3}-F                          \\[2mm]
%     &  \overline{n}        &  \leftarrow  &   n          +   \rho^0  \   =&    D+F                                    \\[2mm]
%     &  \overline{n}        &  \leftarrow  &   n          +   \omega  \   =&    \frac{D}{3}-F                          \\[2mm]
%     &  \overline{\Lambda}  &  \leftarrow  &  \Sigma^0    +   \rho^0  \   =&    -\frac{2D}{\sqrt{3}}          \\[2mm]
%     &  \overline{\Lambda}  &  \leftarrow  &  \Lambda     +   \omega  \   =&    \frac{2D}{3}                  \\[2mm]
% \end{array}$
%%%
 $  \begin{array}{lrr}
\overline{p}p\rho^0              \  & \propto &   -D-F                   \\[2mm]
\overline{p}p\omega              \  & \propto &   \frac{D}{3}-F          \\[2mm]
\overline{n}n\rho^0              \  & \propto &   D+F                    \\[2mm]
\overline{n}n\omega              \  & \propto &   \frac{D}{3}-F          \\[2mm]
\overline{\Lambda}\Lambda\omega  \  & \propto &   \frac{2D}{3}           \\[2mm]
 \end{array}$
 \end{minipage}
 &
\begin{minipage}[c]{0.3\textwidth}
%  $\begin{array}{rclrr}
%    &  \overline{{{\Xi }^-}}   & \leftarrow &   {{\Xi}^-}   +   {{\rho }^0}       \    =&       -D+F \\[2mm]
%    &  \overline{{{\Xi }^-}}   & \leftarrow &   {{\Xi}^-}   +   \omega            \    =&    \frac{D}{3}+F \\[2mm]
%    &  \overline{{{\Xi }^0}}   & \leftarrow &   {{\Xi}^0}   +   {{\rho }^0}       \    =&        D-F \\[2mm]
%    &  \overline{{{\Xi }^0}}   & \leftarrow &   {{\Xi}^0}   +   \omega            \    =&      \frac{D}{3}+F \\[2mm]
%  \end{array}$
%%%
  $\begin{array}{lrr}
\overline{{{\Xi }}^-}{{\Xi}^-}{{\rho }^0}   \   & \propto &    -D+F          \\[2mm]
\overline{{{\Xi }}^-}{{\Xi}^-}\omega        \   & \propto &    \frac{D}{3}+F \\[2mm]
\overline{{{\Xi }}^0}{{\Xi}^0}{{\rho }^0}   \   & \propto &    D-F           \\[2mm]
\overline{{{\Xi }}^0}{{\Xi}^0}\omega        \   & \propto &    \frac{D}{3}+F \\[2mm]
  \end{array}$
\end{minipage}
\end{tabular}
\vspace{1mm}
\hrule
\vspace{1mm}
\hrule
\vspace{10mm}
\end{center}
\end{table}
%
We note however that the couplings in Tables~\ref{tab:FDScouplings}
and \ref{tab:FDcouplings} do not display isospin symmetry manifestly,
though our original Lagrangian density (refer to
Section~\ref{sec:lagrangiandensity}) was constructed in terms of
isospin groups only with common coefficients. This can be remedied by
considering a \emph{general} Lagrangian density constructed from
isospin groups, which we shall restrict to terms involving like
baryons and the mesons we are interested in, to give
%
\bea
%\begin{split}
\nonumber \mathcal{L}_{\rm int}^{\rm oct} &=&
-f_{N\rho}(\overline{N}\vec{\tau}^{\rm T}N)\cdot\vec{\rho}
+if_{\Sigma\rho}(\vec{\overline{\Sigma}}\times\vec{\Sigma})\cdot\vec{\rho}
%-f_{\Lambda\Sigma\rho}(\overline{\Lambda}\mbox{\boldmath $\Sigma$}
%+\overline{\mbox{\boldmath $\Sigma$}}\Lambda)\cdot\vec{\rho}
-f_{\Xi\rho}(\overline{\Xi}\vec{\tau}^{\rm
  T}\Xi)\cdot\vec{\rho} \\
%& -f_{\Lambda NK}[
%(\overline{N}K)\Lambda+\overline{\Lambda}(\overline{K}N) ]
%-f_{\Xi\Lambda K}[
%(\overline{\Xi}K_c)\Lambda+\overline{\Lambda}(\overline{K_c}\Xi) ]
%\\ & -f_{\Sigma NK}[ \overline{\mbox{\boldmath
%$\Sigma$}}\cdot(\overline{K}\mbox{\boldmath $\tau$}^{\rm T}N) +
%(\overline{N}\mbox{\boldmath $\tau$}^{\rm T}K)\cdot\mbox{\boldmath
%$\Sigma$} ] -f_{\Xi\Sigma K}[ \overline{\mbox{\boldmath
%$\Sigma$}}\cdot(\overline{K_c}\mbox{\boldmath $\tau$}^{\rm T}\Xi)
%+(\overline{\Xi}\mbox{\boldmath $\tau$}^{\rm
%T}K_c)\cdot\mbox{\boldmath $\Sigma$} ] \\
\label{eq:octintL}
&& - f_{N\omega}(\overline{N}N)\omega -
f_{\Lambda\omega}(\overline{\Lambda}\Lambda)\omega
-f_{\Sigma\omega}(\vec{\overline{\Sigma}}\cdot\vec{\Sigma})\omega -
f_{\Xi\omega}(\overline{\Xi}\Xi)\omega,
%\end{split}
\eea
%
where the $N$, $\Lambda$, and $\Xi$ isospin groups are defined as
before as
%
\be N = \begin{pmatrix}p\\n\end{pmatrix}, \quad \Lambda
  = \begin{pmatrix}\Lambda\end{pmatrix},
 \quad \Xi = \begin{pmatrix}\Xi^0\\ \Xi^-\end{pmatrix}. \ee
%
%\be
%K = \begin{pmatrix}K^+\\K^0\end{pmatrix} \quad K_c = \begin{pmatrix}\overline{K^0}\\ -K^-\end{pmatrix}
%\ee
%
The $\rho$ mesons terms are defined in isospin space as linear
combinations of the physical charged states (as we did in
Section~\ref{subsec:rotational}) as
%
\be \nonumber \rho^- = \reci{\sqrt{2}}(\rho_1 - i\rho_2), \quad \rho^+
= \reci{\sqrt{2}}(\rho_1 + i\rho_2), \quad \rho^0 = \rho_3, \ee
%
or, equivalently as
%
\be \rho_1 = \reci{\sqrt{2}}(\rho^+ + \rho^-), \quad \rho_2 =
\frac{i}{\sqrt{2}}(\rho^- - \rho^+), \quad \rho_3 = \rho^0, \ee
%
with the same convention for the replacement of $\vec{\rho} \to
\vec{\Sigma}$. This gives the expansion
%
\be \vec{\Sigma}\cdot\vec{\rho} = \Sigma^+\rho^- + \Sigma^0\rho^0 +
\Sigma^-\rho^+ .  \ee
%
We can expand the Lagrangian density term by term to find the
individual interactions
%
\bea \nonumber (\overline{N}\vec{\tau}^{\rm T}N)\cdot\vec{\rho} &=&
(\overline{p} \ \ \overline{n}) \ \tau^{\rm T}_i
\rho^i \begin{pmatrix}p\\n\end{pmatrix} \\ \nonumber &=&
  (\overline{p}n+\overline{n}p)\rho_1 +
  i(\overline{p}n-\overline{n}p)\rho_2 + (\overline{p}p -
  \overline{n}n)\rho_3 \\ \nonumber &=&\reci{\sqrt{2}}
  (\overline{p}n+\overline{n}p)(\rho_+ + \rho_-)
  -\frac{1}{\sqrt{2}}(\overline{p}n-\overline{n}p)(\rho_- - \rho_+) +
  (\overline{p}p - \overline{n}n)\rho_0 \\ &=& \overline{p}p\rho_0 -
  \overline{n}n\rho_0 +\sqrt{2}\overline{p}n\rho_+ +
  \sqrt{2}\overline{n}p\rho_-\ , \eea
%
where we note that a term $\bar{B}BP$ indicates the annihilation of a
baryon $B$ with a meson $P$, and the creation of a baryon $\bar{B}$
according to the reaction $B+P\to\bar{B}$. Continuing to expand terms,
for the $\Sigma$ baryons we have
%
\bea \nonumber
(\vec{\overline{\Sigma}}\times\vec{\Sigma})\cdot\vec{\rho} &=& -i
\rho^+ \left(\Sigma^-\overline{\Sigma^0}-\Sigma^0\overline{\Sigma^+}
\right) -i \rho^- \left(
\Sigma^0\overline{\Sigma^-}-\Sigma^+\overline{\Sigma^0} \right) %\\ &&
-i \rho^0 \left(
\Sigma^+\overline{\Sigma^+}-\Sigma^-\overline{\Sigma^-} \right)\ ,\\[1mm]
&& \eea
%
and for the $\Xi$ baryons,
%
%\be
%( \overline{\Lambda}\mbox{\boldmath $\Sigma$}+\overline{\mbox{\boldmath $\Sigma$}}\Lambda ) 
%\cdot \vec{\rho} = \overline{\Lambda}(\Sigma^+\rho_- + \Sigma^0\rho_0 + \Sigma^-\rho_+)
%+ (\overline{\Sigma^+}\rho_- + \overline{\Sigma^0}\rho_0 + \overline{\Sigma^-}\rho_+)\Lambda
%\ee
%
\bea \nonumber (\overline{\Xi}\vec{\tau}^{\rm T}\Xi)\cdot\vec{\rho}
&=& (\overline{\Xi^0} \ \ \overline{\Xi^-}) \ \tau^{\rm T}_i
\rho^i \begin{pmatrix}\Xi^0\\ \Xi^-\end{pmatrix}\\ \nonumber &=&
  (\overline{\Xi^0}\Xi^-+\overline{\Xi^-}p)\rho_1 +
  i(\overline{\Xi^0}\Xi^--\overline{\Xi^-}\Xi^0)\rho_2 +
  (\overline{\Xi^0}\Xi^0 - \overline{\Xi^-}\Xi^-)\rho_3 \\[2mm] &=&
  \overline{\Xi^0}\Xi^0\rho_0 - \overline{\Xi^-}\Xi^-\rho_0
  +\sqrt{2}\overline{\Xi^0}\Xi^-\rho_+ +
  \sqrt{2}\overline{\Xi^-}\Xi^0\rho_-. \\ \nonumber \eea
%
%\begin{eqnarray}
%\nonumber
%&(\overline{N}K)\Lambda+\overline{\Lambda}(\overline{K}N) 
%&= ( \overline{p} \ \ \overline{n} ) \begin{pmatrix}K^+\\K^0\end{pmatrix}\Lambda 
%+ \overline{\Lambda}( \overline{K^+} \ \ \overline{K^0} )\begin{pmatrix}p\\n\end{pmatrix}\\
%&&=\overline{p}K^+\Lambda + \overline{n}K^0\Lambda 
%+ \overline{\Lambda}\overline{K^+}p + \overline{\Lambda}\overline{K^0}n
%\end{eqnarray}
%
%\begin{eqnarray}
%\nonumber
%&(\overline{\Xi}K)\Lambda+\overline{\Lambda}(\overline{K}\Xi) 
%&= ( \overline{\Xi^0} \ \ \overline{\Xi^-} ) \begin{pmatrix}\overline{K^0}\\-K^-\end{pmatrix}\Lambda 
%+ \overline{\Lambda}( K^0 \ \ -\overline{K^-} )\begin{pmatrix}\Xi^0\\\Xi^-\end{pmatrix}\\
%&&=\overline{\Xi^0}\overline{K^0}\Lambda - \overline{\Xi^-}K^-\Lambda 
%+ \overline{\Lambda}K^0\Xi^0 - \overline{\Lambda}\overline{K^-}\Xi^-
%\end{eqnarray}
%
%\begin{eqnarray}
%\nonumber
%& \overline{\mbox{\boldmath $\Sigma$}}\cdot(\overline{K}\mbox{\boldmath $\tau$}^{\rm T}N) 
%+ (\overline{N}\mbox{\boldmath $\tau$}^{\rm T}K)\cdot\mbox{\boldmath $\Sigma$}
%&=(\overline{K^+} \ \ \overline{K^0}) \ \tau^{\rm T}_i \Sigma^i \begin{pmatrix}p\\n\end{pmatrix}
%+ (\overline{p} \ \ \overline{n})  \ \tau^{\rm T}_i \Sigma^i \begin{pmatrix}K^+\\K^0\end{pmatrix} \\
%\nonumber
%&&= \sqrt{2}\ \overline{K^0}p\overline{\Sigma^+} + \sqrt{2}\ \overline{K^+}n\overline{\Sigma^-} 
%+ \overline{K^+}p\overline{\Sigma^0} - \overline{K^0}n\overline{\Sigma^0} \\
%&&+ \sqrt{2}\ K^0\overline{p}\Sigma^+ + \sqrt{2}\ K^+\overline{n}\Sigma^- 
%+ K^+\overline{p}\Sigma^0 - K^0\overline{n}\Sigma^0
%\end{eqnarray}
%
%\begin{eqnarray}
%\nonumber
%& \overline{\mbox{\boldmath $\Sigma$}}\cdot(\overline{K_c}\mbox{\boldmath $\tau$}^{\rm T}\Xi) 
%+ (\overline{\Xi}\mbox{\boldmath $\tau$}^{\rm T}K_c)\cdot\mbox{\boldmath $\Sigma$}
%&=(K^0 \ -\overline{K^-}) \ \tau^{\rm T}_i \Sigma^i \begin{pmatrix}\Xi^0\\ \Xi^-\end{pmatrix}
%+ (\overline{\Xi^0} \ \ \overline{\Xi^-})  \ \tau^{\rm T}_i \Sigma^i \begin{pmatrix}\overline{K^0}\\-K^-\end{pmatrix} \\
%\nonumber
%&&= \sqrt{2}\ \overline{K^-}\Xi^0\overline{\Sigma^+} + \sqrt{2}\ \overline{K^0}\Xi^-\overline{\Sigma^-} 
%+ \overline{K^0}\Xi^0\overline{\Sigma^0} - \overline{K^-}\Xi^-\overline{\Sigma^0} \\
%&&+ \sqrt{2}\ K^-\overline{\Xi^0}\Sigma^+ + \sqrt{2}\ K^0\overline{\Xi^-}\Sigma^-
%+ K^0\overline{\Xi^0}\Sigma^0 - K^-\overline{\Xi^-}\Sigma^0
%\end{eqnarray}
%
%Now the diagonals:
%
The iso-scalar terms are more straightforward;
%
\bea (\overline{N}N)\omega &=& \overline{p}p\omega +
\overline{n}n\omega, \\[1mm]
%
\overline{\Lambda}\Lambda\omega && (\textrm{requires no expansion}), \\[1mm]
%
( \vec{\overline{\Sigma}}\cdot\vec{\Sigma})\omega &=&
\overline{\Sigma^+}\Sigma^+\omega + \overline{\Sigma^0}\Sigma^0\omega
+ \overline{\Sigma^-}\Sigma^-\omega, \\[1mm]
%
(\overline{\Xi}\Xi)\omega &=& \overline{\Xi^0}\Xi^0\omega +
\overline{\Xi^-}\Xi^-\omega.  \eea
%
\par
%
Once we have calculated the full interaction Lagrangian density, and
the $F$ and $D$ coefficients of each interaction, we have factors of
the following form:
%
\be {\cal L}_{\rm int} = \sum_B \sum_m A_{Bm} f_{Bm} X_{Bm}; \quad
X_{Bm} = C_{Bm} \bar{B}Bm, \ee
%
where $A_{\Sigma \rho} = i$, and for all other interactions $A_{Bm}=
-1$. The term $X_{Bm}$ is the expanded interaction term (after
expanding cross products, etc.) arising from the \emph{general}
Lagrangian density, Eq.~(\ref{eq:octintL}), and contains factors of
$C_{Bm} = \pm 1,\pm \sqrt{2}$. We also require a term calculated from
the SU(3) invariant combinations $M_{Bm}$; the coefficient of the
interaction $\bar{B}Bm$ in terms of $F$ and $D$ factors as found in
Table~\ref{tab:FDcouplings}.\par
%
To calculate the values of $f_{Bm}$ we apply the following formula:
%
\be f_{Bm} = \frac{C_{Bm}}{A_{Bm}} M_{Bm}. \ee
%
For example, consider the interaction vertex $\omega + \Sigma^0 \to
\overline{\Sigma^0}$:
%
\be A_{\Sigma \w} = -1, \quad C_{\Sigma \w} = +1, \quad M_{\Sigma \w}
= -\frac{2D}{3}, \quad \Rightarrow \quad f_{\Sigma\omega} =
\frac{+1}{-1}(-\frac{2D}{3}) = \frac{2D}{3}.  \ee
%
Performing these calculations for every possible interactions provides
(consistently) the following couplings of the octet of baryons to the
octet of mesons:\par
%
\be 
\begin{array}{c}
f_{N\rho} = D+F,\quad f_{\Lambda\rho} = 0,\quad f_{\Sigma\rho} =
2F, \quad f_{\Xi\rho} = F-D, %\ee
%
\\[2mm]
%\be 
f_{N\w} = 3F-D, \quad f_{\Lambda\w} = -\frac{4}{3}D+2F, \quad
f_{\Sigma\w} = 2F, \quad f_{\Xi\w} = F-D. 
\end{array}
\ee
%
\par
%
We can further simplify our calculations by examining all the
different currents that one can form using a baryon, an antibaryon and
a meson. As discussed in Ref.~\cite{Sakita:1965qt}, all possible
couplings of baryons to vector mesons (denoted by $\bar{B}BV$) should
be considered when writing out the most general Lagrangian density. By
calculating the currents (prior to making any approximations or
assumptions that appear in earlier sections here) we are able to find
the $F$-, $D$-, and $S$-style couplings of the form $\bar{B}BX$ where
$X$ is a meson with either scalar (S), vector (V), tensor (T),
axial-vector (A) or pseudo-scalar (P) spin form.\par
%
Under an expanded SU(6) spin-flavor symmetry, the currents are shown
in Table~\ref{tab:currents}, where the various vector couplings are of
the forms
%
\be V_1 = \bar{\psi}\gamma_\mu\psi,\ V_2 =
\bar{\psi}\sigma_{\mu\nu}q^\nu\psi,\ V_3 = \bar{\psi}q_\mu\psi, \ee
%
and we use the convenience definitions of
%
\be \sigma_{\mu\nu}=\frac{i}{2}\left[\gamma_\mu,\gamma_\nu\right], \quad
H=\frac{4M^2+q^2}{2M^2}.  \ee
%
\par
%
\begin{table}[!t]
\centering
\caption[$F$-, $D$- and $S$-style baryon currents]{\protect $F$-, $D$-
  and $S$-style Baryon currents for all types of meson vertices of the
  form $\bar{B}BX$ where $X$ is a meson with either scalar (S), vector
  (V), tensor (T), axial(pseudo-) vector (A) or pseudo-scalar (P) spin
  form. Adapted from Ref.~\cite{Sakita:1965qt}.\label{tab:currents}}
\vspace{3mm}
\begin{tabular}{lccc}
\hline 
\hline 
&&&\\[-4mm]
   &     $F$    &    $D$    &    $S$ \\
&&&\\[-4mm]
\hline
&&&\\[-2mm]
S  & $\reci{3}H\bar{\psi}\psi$ & 0 & $\reci{3}H\bar{\psi}\psi$ \\
&&&\\[-2mm]
V${}_1$ & $\reci{3}\left(H-\reci{6}\frac{q^2}{M^2}\right)\bar{\psi}\gamma_\mu\psi$ &
 $\frac{q^2}{6M^2}\bar{\psi}\gamma_\mu\psi$ &
 $\reci{3}\left(H-\reci{3}\frac{q^2}{M^2}\right)\bar{\psi}\gamma_\mu\psi$ \\
&&&\\[-2mm]
V${}_2$ & $-\reci{9M}i\bar{\psi}\sigma_{\mu\nu}\psi$ &  
$ + \reci{3M}i\bar{\psi}\sigma_{\mu\nu}\psi$ &
$ - \frac{2}{9M}i\bar{\psi}\sigma_{\mu\nu}\psi$ \\
&&&\\[-2mm]
V${}_3$ &  0  &  0  &  0  \\
&&&\\[-2mm]
A & $ \frac{2}{9}H\bar{\psi}\gamma_5\gamma_\mu\psi $ &
    $ \reci{3}H\bar{\psi}\gamma_5\gamma_\mu\psi $    & 
    $ \reci{9}H\bar{\psi}\gamma_5\gamma_\mu\psi $ \\
&&&\\[-2mm]
P & $ \frac{2}{9}H\bar{\psi}\gamma_5\psi $ &
    $ \reci{3}H\bar{\psi}\gamma_5\psi $    & 
    $ \reci{9}H\bar{\psi}\gamma_5\psi $ \\
&&&\\[-2mm]
\hline
\hline
\end{tabular}
\end{table}
%
If we now consider this as a low energy effective field theory, we can
consider the case of $q^2=0$. We can also enforce rotational symmetry
due to lack of a preferred frame (or direction) and thus remove the
spatial components of both the mesons and the momenta, so that
$V^\mu=(V^0,\vec{0})$ and $q^\mu=(q^0,\vec{0})$, as per
Section~\ref{subsec:rotational}. Along with $\sigma_{00}=0$, all terms
proportional to $q^2$ vanish, and $H=2$. Using these assumptions, the
currents are reduced to those found in
Table~\ref{tab:currentassumptions}.\par
%
We can now observe the relations between the $F$- and $D$-style
couplings (with the $S$-style coupling now contributing to $F$ and
$D$); First, as a check, we observe that the ratio $D/F$ for the
pseudo-scalars (and the axial-vectors for that matter) is indeed
$\frac{3}{2}$ as commonly noted in the
literature~\cite{Gursey:1964,Aliev:2001} under SU(6)
symmetry~\cite{Ishida:1968}. Less commonly found in the literature is
that the $\gamma_\mu$-type vector coupling is purely $F$-style, thus
the vector analogy of the above relation is $D/F=0$, implying
$D=0$.\par
%
Using the couplings of Table~\ref{tab:FDcouplings}, we can evaluate
the couplings of the vector mesons to the entire baryon octet. This
provides us with a unified description of the couplings in terms of an
arbitrary parameter $F$. These couplings are thus
%
\be%a
\begin{array}{c}
%\nonumber
f_{N\rho} = F,\quad f_{\Lambda\rho} = 0,\quad f_{\Sigma\rho} =
2F,\quad f_{\Xi\rho} = F, \\[2mm] f_{N\w} = 3F,\quad f_{\Lambda\w} =
2F,\quad f_{\Sigma\w} = 2F,\quad f_{\Xi\w} = F.
\end{array}
\ee%a
%
We can normalize these results to the nucleon-$\w$ coupling, since we
will fit this parameter to saturation properties (refer to
Section~\ref{sec:EoS}). Thus the normalized couplings are
%
\be g_{Bm} = g_{N\w}\frac{f_{Bm}}{f_{N\w}}. \ee
%
We can then separate the meson couplings, since the the normalization
above results in the following relations, using isospin $I_B$, and
strangeness $S_B$ of baryon $B$;
%
\be g_{B\w} =\ \frac{(3-S_B)}{3}\ g_{N\w}, \quad g_{B\rho}
=\ \frac{2I_B}{3}\ g_{N\w}.  \ee
%
These results are consistent with a commonly used na\"ive assumption
that the $\w$ meson couples to the number of light quarks, and that
the $\rho$ meson couples to isospin. To emphasize the isospin symmetry
in our models, we will include the isospin as a factor in our
Lagrangian densities in the form of the $\vec{\tau}$ matrices. In
doing so, rather than having an independent coupling for each isospin
group, we will have a global coupling for the $\rho$ meson,
$g_{\rho}$. \par
%
\begin{table}[!t]
\centering
\caption[$F$-, $D$- and $S$-style baryon currents with mean-field
  assumptions]{\protect $F$-, $D$- and $S$-style Baryon currents with
  mean-field assumptions $V^\mu=(V^0,\vec{0})$, $q^\mu=(q^0,\vec{0})$,
  and $q^2=0$. \label{tab:currentassumptions}}
\vspace{3mm}
\begin{tabular}{lccc}
\hline 
\hline 
&&&\\[-4mm]
   &     $F$    &    $D$    &    $S$ \\
&&&\\[-4mm]
\hline
&&&\\[-2mm]
S  & $\frac{2}{3}\bar{\psi}\psi$ & 0 & $\frac{2}{3}\bar{\psi}\psi$ \\
&&&\\[-2mm]
V${}_1$ & $\frac{2}{3}\bar{\psi}\gamma_\mu\psi$ & 0 & $\frac{2}{3}\bar{\psi}\gamma_\mu\psi$ \\
&&&\\[-2mm]
V${}_2$ & 0 & 0 & 0 \\
&&&\\[-2mm]
V${}_3$ &  0  &  0  &  0  \\
&&&\\[-2mm]
A & $ \frac{4}{9}\bar{\psi}\gamma_5\gamma_\mu\psi $ &
    $ \frac{2}{3}\bar{\psi}\gamma_5\gamma_\mu\psi $    & 
    $ \frac{2}{9}\bar{\psi}\gamma_5\gamma_\mu\psi $ \\
&&&\\[-2mm]
P & $ \frac{4}{9}\bar{\psi}\gamma_5\psi $ &
    $ \frac{2}{3}\bar{\psi}\gamma_5\psi $    & 
    $ \frac{2}{9}\bar{\psi}\gamma_5\psi $ \\
&&&\\[-2mm]
\hline
\hline
\end{tabular}
\end{table}
%
Similarly to the above relations for the vector mesons, we have the
same relation for the scalar mesons; that the coupling is purely
$F$-style ($D=0$). Therefore the couplings for the scalar mesons are
the same as for the vector mesons, under the replacements $\w \to \s$,
$\vec{\rho} \to \vec{a}_0$. In the calculations that follow, we shall
further neglect the contributions from the scalar iso-vector
$\vec{a}_0$ due to their relatively large mass (refer to
Table~\ref{tab:particlesummary}).\par
%
%While we have derived the above relation for the $\rho$ meson, it
%relies on the SU(3)$\ocross$SU(2) symmetry, which isn't a perfect symmetry
%considering the mass of the strange quark compared to the light
%quarks. As an improvement to the models, 
As an alternative to the SU(6) relations for the $\rho$ meson coupling
$g_\rho$, we can use an experimental constraint. As we have shown
above, the $\rho$ meson couples to isospin, and as we will show in
Section~\ref{sec:qhd} the isospin density is proportional to the
asymmetry between members of an isospin group; for example the
asymmetry between protons and neutrons. This asymmetry is measured by
the symmetry energy $a_4 \equiv a_{\rm sym}$ (derived in
Appendix~\ref{sec:symenergy}) which appears in the semi-empirical mass
formula (the connection is derived in Appendix~\ref{sec:SEMF}) which
in the absence of charge symmetry is defined by Eq.~(\ref{eq:a4}).
%as
%
%\be \label{eq:a4def}
%a_{\rm sym} = \frac{g_{N\rho}^2\; k_{F_{\rm sat}}^3}{3\pi^2m_\rho^2}
%+ \frac{k_{F_{\rm sat}}^2}{12\sqrt{k_{F_{\rm sat}}^2 + (M_{n}^{* 2})_{\rm sat}}}
%+ \frac{k_{F_{\rm sat}}^2}{12\sqrt{k_{F_{\rm sat}}^2 + (M_{p}^{* 2})_{\rm sat}}}.
%\ee
%
The coupling of $\rho$ to the nucleons is found such that the
experimental value of the asymmetry energy of $a_{\rm sym} = 32.5~{\rm
  MeV}$ is reproduced at saturation. The coupling of $\rho$ to the
remaining baryons follows the relations above.\par
%
% \cleardoublepage 

\end{document}
